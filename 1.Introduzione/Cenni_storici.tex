\documentclass[main.tex]{subfiles}

\begin{document}

\subsection{Cenni storici sull'Erbario dell'Università di Bologna}\label{Cap. 1.1}

L’Erbario dell’Università di Bologna \href{https://sma.unibo.it/it/il-sistema-museale/orto-botanico-ed-erbario/}{(L’orto botanico e l’erbario si presentano, 2022)} è da considerarsi come uno dei più antichi d’Europa. Realizzato nel XVI secolo, simultaneamente all’Orto Botanico, per volere di Ulisse Aldrovandi, inizialmente aveva una funzione eminentemente didattica, legata in particolare all’insegnamento della botanica medica poiché consentiva agli studenti lo studio e l’esercitazione su piante officinali reali indipendentemente dalla stagione vegetativa.
Le collezioni di piante essiccate tutt’oggi conservate, furono raccolte a partire dal XVI secolo e, sebbene tale Erbario sia da considerarsi di medio-piccole dimensioni poiché contiene circa 133mila campioni, è lecito affermare che alcune delle sue collezioni storiche realizzate da eminenti direttori dell’Orto Botanico siano tra le più datate del continente europeo.
Il 1500 comportò profonde evoluzioni del pensiero scientifico e Bologna fu uno dei centri nei quali si sviluppò, grazie all’operato di Luca Ghini (1490-1556), la tecnica di realizzare erbari costituiti da vere piante essiccate \citep{von}. Nonostante ciò, il suo erbario personale è da considerarsi perduto.
L’opera di Ulisse Aldrovandi, allievo dello stesso Ghini, è di fatto uno dei più antichi e ampi erbari del suo tempo giunti fino ai giorni nostri. Probabilmente realizzato a partire dal 1551 arrivò a comprendere oltre 5000 campioni suddivisi in 15 volumi rilegati, i quali contengono ognuno centinaia di piante essiccate incollate su fogli \citep{mira}. Per quanto concerne le informazioni relative al cartellino è presente soltanto il nome attribuito alla pianta da parte di Aldrovandi stesso e in casi rari anche il luogo di provenienza e il nome del raccoglitore. Apparentemente la distribuzione delle piante nell’Erbario non sembra seguire alcun criterio sistematico, ad eccezione di quello alfabetico applicato ai primi volumi. Tale raccolta costituisce l’opera di maggior pregio dell’Erbario di Bologna ed il suo valore è dato dal gran numero di piante che lo compongono, dalla sua antichità e dalla cura con cui è stato allestito e preservato.
Nei secoli a venire l’erbario divenne un sistema di collezione sempre più diffuso tanto che alcune collezioni, sebbene siano ancora oggi conservate a Bologna ed ampiamente studiate, restano senza autore certo. Ne sono un esempio un Erbario anonimo datato tra la fine del XVI ed inizio XVII secolo ed un volume d’erbario la cui rilegatura è stata ottenuta utilizzando un manoscritto medievale. Spesso i collaboratori non venivano citati con nome completo o in alternativa venivano nominati tramite pseudonimi, tutto ciò rende difficoltosa l’attribuzione di un nominativo certo.
Nel XIX secolo i viaggi di esplorazione geografica e il crescente studio delle flore nazionali consentono il massimo sviluppo degli erbari e al contempo, nel panorama bolognese ed italiano, Antonio Bertoloni si afferma come il più famoso botanico dell’800. Alcune delle raccolte del periodo sono presenti nell’Erbario di Bologna e rappresentano un patrimonio di indubbio valore.
La sua opera maggiore, la \textit{Flora Italica} \citep{moscri} (in lingua latina il titolo completo è “\textit{Flora italica sistens plantas in Italia et in insulis circumstantibus sponte nascentes}”) fu realizzata prima dell’unificazione politica dell’Italia ed è costituita da 10 volumi. Essa rappresenta una notevole documentazione floristica sistematica relativa alle piante spontanee dell’odierno territorio italiano ed è oggetto di ricerca in questa analisi di tesi per quanto concerne l’Erbario Storico dell’Università di Bologna. Le specie descritte, studiate ed erborizzate furono raccolte da Bertoloni stesso oppure inviategli da altri orti botanici italiani e sono comprensive di indicazioni sulle località nelle quali crescevano. Tali essenze sono conservate nel’”\textit{Hortus Siccus Florae Italicae}” che costituisce una delle collezioni di exsiccata italiane più importarti sia dal punto di vista storico che scientifico, andando a costituire il nucleo principale dell’Erbario bolognese. Per ogni esemplare nel rispettivo cartellino è riportato il nome scientifico della pianta, la data e la località di raccolta, il nome del raccoglitore e il riferimento al volume e alla pagina in cui viene descritta.
Nei primi volumi Bertoloni adotta la nomenclatura binomia e organizza i campioni secondo il sistema di classificazione Linneano. La nomenclatura binomia, tutt’ora in uso, impone di assegnare ad ogni organismo due nomi univoci, uno per il genere e uno per la specie, al fine di sostituire le equivoche definizioni polinomiali. Il sistema classificatorio Linneano prevede invece l’ordinamento delle specie in base agli organi della riproduzione sessuata, ovvero sulla morfologia di stami e pistilli.
Antonio Bertoloni non fu solo un grande studioso della flora italica, ma descrisse anche numerose specie esotiche. I campioni inviatigli dai suoi corrispondenti sono oggi raccolti nella collezione da lui stesso denominata \textit{Hortus Siccus Exoticus} \href{https://sma.unibo.it/it/il-sistema-museale/orto-botanico-ed-erbario/collezioni/antonio-bertoloni-hortus-siccus-exoticus-erbario}{(Antonio Bertoloni: Hortus Siccus Exoticus, 2022)}. Essa include più di 11.000 campioni, per un totale di 139 famiglie e 1544 generi provenienti da varie aree geografiche come Europa, America Centrale, Medio Oriente e parti del continente asiatico quali Siberia e India. Alcuni campioni furono utilizzati nella compilazione della Flora Italica, ma la maggior parte vengono descritti in altre opere di Bertoloni, come le \textit{Miscellanee Botaniche} oppure la \textit{Florula Guatimalensis} \citep{band}.
Quest’ultima raccolta risulta essere particolarmente interessante perché contiene olotipi della flora del Guatemala, che venne studiata e descritta per primo da Bertoloni stesso. Il botanico bolognese entrò in possesso di questa piccola raccolta di piante, costituita da 79 campioni essiccati, nel 1836 a Bologna quando una delegazione messicana diretta a Roma per una visita al Pontefice vi sostò e l’ufficiale Joachim Velasquez, che nonostante la carriera militare era un grande appassionato di scienze naturali, donò a Bertoloni tale collezione di piante essiccate. I campioni erano nella maggior parte dei casi sconosciuti alla comunità scientifica e successivamente furono oggetto di numerose pubblicazioni da parte di Bertoloni.
Molteplici altre raccolte degne di nota realizzate da altri botanici sono conservate a Bologna e in particolare, sebbene non siano parte integrante di questa analisi risultano essere degne di nota, quali: l’Erbario di Giuseppe Monti (1682-1760) e quello di Ferdinando Bassi (1710-1774). I loro contenuti biologici furono riclassificati secondo il sistema linneano da Antonio Bertoloni stesso, poiché riportavano soltanto il nome della pianta e una descrizione facente riferimento ad opere obsolete. Entrambe le raccolte contengono esemplari provenienti dall’Erbario di Ulisse Aldrovandi. Altri erbari presenti invece furono realizzati da studenti universitari di Giuseppe Monti e del figlio Gaetano Lorenzo (1712-1797), come i 2 volumi contenenti 222 campioni medicinali di Stefano Bartolotti, oltre a 3 volumi di un \textit{Herbarium Medicum}, ovvero raccolte di piante medicinali di incerta attribuzione e datazione. Inoltre, vi sono alcune raccolte terminate o interamente realizzate da Giuseppe Bertoloni, (figlio di A. Bertoloni, 1804-1878), come l’\textit{Hortus Siccus Florae Bononiensis} e \textit{Hortus Siccus Plantarum Medicinalis}. Infine, vi è da menzionare la più recente raccolta realizzata da Emilio Chiovenda (1871-1941), il quale si dedicò in maniera approfondita allo studio della flora africana, oltre che di quella italiana, collaborando alla stesura della Flora della Colonia Eritrea.
Nonostante in origine l’\textit{Hortus Siccus Florae Italicae} \href{https://sma.unibo.it/it/il-sistema-museale/orto-botanico-ed-erbario/collezioni/antonio-bertoloni-flora-italica-e-hortus-siccus-florae-italicae}{(Antonio Bertoloni: Flora Italica E Hortus Siccus Florae Italicae, 2022)} comprendesse 803 generi e 4211 specie vegetali, eventi bellici e soprattutto incuria e disinteresse per la sua tutela hanno portato alla perdita di quasi un quarto della collezione. Nonostante ciò, continua a rappresentare una rilevante porzione dell’Erbario bolognese stesso, poiché continua a essere un più che valido strumento di studio.
L’Erbario bolognese, nonostante le importanti collezioni storiche che detiene, è attivo nel ricercare e accogliere nuovi campioni provenienti da ricerche scientifiche, poiché tali acquisizioni possono fornire dati importanti per lo studio della distribuzione spaziale e temporale delle specie in un determinato territorio e della biodiversità vegetale a livello regionale e nazionale.


\end{document}