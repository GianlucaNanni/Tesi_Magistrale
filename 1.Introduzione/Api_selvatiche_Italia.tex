\documentclass[main.tex]{subfiles}

\begin{document}

\subsection{Le api selvatiche in Italia e i progetti di comunicazione e conservazione}

Le specie di api selvatiche europee sono circa duemila e di queste la metà è presente sul territorio italiano, facendo di esso uno dei più ricchi in Europa per biodiversità. Nella regione mediterranea, l’Italia presenta un’elevata biodiversità grazie alla latitudine e al gradiente altitudinale della penisola, che abbinato alla complessità geologica e orografica, determina una grande diversità di climi e ambienti naturali. Inoltre, la collocazione geografica dell’Italia al centro del bacino del Mediterraneo comporta la presenza di specie derivanti da diverse sotto-regioni biogeografiche limitrofe, rendendo la fauna apistica italiana una tra le più ricche del mondo in rapporto alla superficie del Paese. Gli apoidei selvatici presenti in Italia sono rappresentati da 6 famiglie (Andrenidae, Apidae, Colletidae, Halictidae, Megachilidae, Melittidae) contenenti circa 58 generi che annoverano nel complesso 944 specie \href{https://www.wwf.it/pandanews/ambiente/impollinatori-ditalia/}{(Impollinatori d’Italia, 2020)}.
Riguardo la distribuzione delle specie selvatiche nel nostro Paese si hanno davvero poche informazioni, peraltro non esaustive, provenienti dalle principali aree protette nazionali.
Sebbene accertare la presenza di una determinata specie in un’area possa rilevarsi essenziale per la sua conservazione, ad oggi le informazioni disponibili sono frammentarie ed inoltre non vi è alcun riferimento numerico che consenta di effettuare una valutazione oggettiva delle tendenze demografiche dell’insieme delle popolazioni italiane. Dunque, è necessario sia effettuare nuovi e costanti campionamenti ad opera di personale opportunamente addestrato, che ricercare informazioni relative alla presenza di specie contenute in collezioni di Apoidei che ancora non sono state digitalizzate.
A tal proposito l’Unione Internazionale per la Conservazione della Natura (IUCN, International Union for Conservation of Nature), organizzazione finalizzata alla conservazione dell’integrità e diversità della natura e all’utilizzo equo ed ecologicamente sostenibile delle risorse naturali, ha redatto una Lista Rossa IUCN delle specie di api italiane minacciate \citep{quara}.\\
La realizzazione di una Lista Rossa degli Apoidei ha difatti lo scopo di focalizzare l’attenzione su un gruppo di specie la cui conservazione è di grande importanza. Gli obiettivi sono di selezionare le specie ritenute a maggior rischio di estinzione e identificare le azioni atte a contrastare le principali minacce e consentirne la conservazione. Tale elenco, contenente 151 specie in totale, comprende 85 specie dichiarate minacciate dal team di esperti europeo ed ulteriori 66 specie la cui valutazione su basa sulla rarità, scarsa abbondanza o presenza marginale in Italia.
Al fine di comprendere ed intraprendere azioni di salvaguardia per gli impollinatori selvatici sono in atto, in Italia e nel mondo, molteplici iniziative sostenute da governi, organizzazioni internazionali e istituzioni scientifiche. Di seguito ne sono riportati alcuni modelli.\\
Ne è un esempio il recente progetto della Commissione europea “Safeguarding European wild pollinators”, progetto iniziato nel 2021 e dalla durata di quattro anni, avente l’obiettivo di trovare nuovi approcci per preservare gli impollinatori selvatici europei, per comprendere i complessi fattori che interagiscono tra loro determinando un calo della biodiversità, per aumentare la consapevolezza sociale sul valore delle api con il pubblico e mobilitare azioni finalizzate ad invertire il declino degli impollinatori in tutta Europa \href{https://cordis.europa.eu/project/id/101003476/it}{(Safeguarding European wild pollinators, 2022)}.\\
Il progetto SPRING (Strengthening Pollinator Recovery through Indicators and monitorinG) viene finanziato dall’UE, è iniziato a maggio 2021 e proseguirà fino a novembre 2023. Tale programma si prefigge di creare un sistema di monitoraggio scientificamente solido e sostenibile degli impollinatori presenti in Europa tramite indicatori in grado di rilevare cambiamenti significativi nell’abbondanza degli impollinatori \href{https://www.ufz.de/spring-pollination/}{(SPRING, 2022)}.\\
Il progetto LIFE4Pollinators, coordinato dall'Università di Bologna e cofinanziato dal fondo europeo LIFE, è finalizzato a migliorare la conservazione degli insetti impollinatori e delle piante entomofile tramite lo sviluppo di Protocolli Comuni, programmi di Citizen Science e la creazione di reti virtuose finalizzate al cambiamento di pratiche antropiche dannose per gli insetti impollinatori \href{https://www.life4pollinators.eu/}{(LIFE 4 Pollinators, 2022)}.\\
Il progetto LIFE PollinAction, condotto dall'Università Ca' Foscari di Venezia, anch’esso cofinanziato dall’Unione Europea con fondi Life, è finalizzato a potenziare il ruolo degli impollinatori in ambienti rurali e urbani. Nei paesaggi altamente semplificati, agricoli intensivi o urbani che siano, l’obiettivo è di assicurare habitat idonei alle esigenze degli impollinatori tramite la creazione di una Green Infrastructure costituita da aree naturali e semi-naturali \href{https://mizar.unive.it/lifepollinaction.eu/}{(Life PollinAction, 2021)}.\\
Il progetto di Citizen Science X-Polli:Nation è nato grazie a un finanziamento del National Geographic USA, è supportato dal Tuscany Envinronment Foundation e i partner universitari Toscani (Firenze, Pisa e Siena) coinvolti sono coordinati dal Museo di Storia Naturale della Maremma. Lo scopo che si prefigge tale progetto è di monitorare le interazioni tra piante e insetti impollinatori, raccogliendo dati su preferenze alimentati e habitat naturali nei due differenti paesi (Italia e Regno Unito) in cui è attivo. Questa indagine viene sviluppata coinvolgendo scienziati, educatori e studenti \href{https://xpollination.org/}{(X-Polli:Nation, 2019)}.\\
Il progetto ORBIT viene finanziato dalla Commissione Europea, ha durata di 3 anni ed è attualmente in atto. Lo scopo che si prefigge è di sviluppare una struttura tassonomica centralizzata che funga da base per l'identificazione delle api selvatiche europee e da supporto ad altri progetti europei finalizzati allo studio e alla tutela degli impollinatori \href{https://orbit-project.eu/}{(Bee diversity in Europe, 2021)}.

\end{document}