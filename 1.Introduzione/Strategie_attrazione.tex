\documentclass[main.tex]{subfiles}

\begin{document}

\subsection{Strategie di attrazione delle api da parte delle piante}

L’evoluzione degli organismi animali e vegetali ha come finalità la sopravvivenza della specie e la simbiosi mutualistica è una strategia che si contrappone alla competizione per le risorse, incentivando un’alleanza tra le parti. La porzione di flora che attua tale strategia è definita entomogama ed è il risultato della coevoluzione che è avvenuta tra piante e fauna impollinatrice.
Le strutture fiorali delle piante presentano particolari forme e di conseguenza selezionano i propri visitatori in basse alla taglia e/o alla forza. In alcuni casi la corolla del fiore viene “forzata” da insetti robusti per raggiungere i nettàri, mentre in altri vi è un meccanismo di rilascio del polline “a bilanciere”, che si attiva solo quando sono visitati da api di taglia elevata, oppure “a scatto”, come nelle Leguminosae-Fabaceae dove i petali inferiori del fiore sono fusi tra loro a formare una piattaforma d’appoggio (la carena) che al suo interno contiene le antere; quando l’insetto si posa sulla carena, le antere escono di scatto liberando il polline \citep{serin}. In alcuni fiori il rilascio può essere molto energico, pertanto vengono visitati solo da pronubi robusti, come bombi (\textit{Bombus}) e silocope (\textit{Xylocopa}).
Di seguito (Tab. \ref{tab:1}) sono riportati i principali legami tra gruppi di Apoidei e famiglie di piante di interesse agrario:\\

\begin{table}[h!]
    \centering
\begin{tabular}{|p{5cm}|p{5cm}|p{5cm}|}
\hline
Generi di Apoidei & Caratteristiche degli impollinatori & Famiglie di piante coltivate\\
\hline
Ammobatoides, Andrena, Anthidium, Apis, Bombus, Eucera, Lasioglossum, Megachile & Api di grosse dimensioni in grado di reggere lo scatto delle antere & Fabaceae\\
\hline
Andrena, Anthidium, Bombus, Osmia & Api di grosse dimensioni capaci di azionare il meccanismo a bilanciere del fiore & Lamiaceae\\
\hline
Eucera, Bombus & Api massicce capaci di forzare l’apertura fiorale & Scrophulariaceae\\
\hline
Apis, Bombus, Eucera & Pronubi di media e grande taglia & Solanaceae\\
\hline
\end{tabular}
    \caption{relazioni tra Apoidei e piante coltivate.}
    \label{tab:1}
\end{table}

\vspace{5cm}

Inoltre, in aggiunta agli attraenti primari già evidenziati nel Cap. \ref{Cap. 1.7}, la flora entomogama possiede attraenti secondari che svolgono una funzione di attrazione visiva per i pronubi, attraverso fiori appariscenti o profumi accattivanti.
Gli insetti possono percepire la riflessione della luce a lunghezze d’onda pari a 300-400 nm, funzione particolarmente importante nell’evitare confusione durante il trasferimento del polline, che deve avvenire il più possibile all’interno della stessa specie \citep{dg}.
Inoltre, vi è un ulteriore ed importante meccanismo utile all’individuazione e al riconoscimento dei fiori che è formato dall’insieme dei profumi volatili emessi composti da benzenoidi e terpenoidi vegetali.
Il rapporto pianta-insetto è un avvicendarsi di interazioni che deve consentire determinati spazi temporali di sincronia per poter essere utile ai soggetti interagenti. Ma le azioni antropiche e i cambiamenti climatici hanno condotto modifiche della fenologia delle specie, inclusi i periodi di sincronia tra le parti, con ripercussioni a cascata sulle comunità e sugli ecosistemi \citep{laz}.
In aggiunta, il sistema pianta-impollinatore può manifestare un differente grado di specializzazione a seconda della struttura fiorale in oggetto. I fiori zigomorfi (come quelli di Fabaceae e Lamiaceae), caratterizzati da una corolla costituita da elementi disposti specularmente su un solo piano di simmetria, sono prevalentemente visitati da apoidei a ligula lunga poiché i nettàri sono ubicati all’interno del calice, mentre i fiori attinomorfi (es. \textit{Prunus}-Rosaceae), aventi corolla simmetrica rispetto a un punto centrale, presentano minore specializzazione e sono visitati da un vasto assortimento di impollinatori \citep{neal}.

\end{document}