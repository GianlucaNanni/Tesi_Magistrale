\documentclass[main.tex]{subfiles}

\begin{document}

\subsection{Monitoraggio delle api selvatiche}

Il monitoraggio delle api selvatiche è un valido metodo da cui trarre indicazioni sullo stato dell’ambiente, sulla qualità ecologica e sull’esistenza di eventuali fonti di inquinamento.
Gli apoidei sono animali ubiquitari e spesso vivono stabilmente nell’ecosistema di interesse, in aggiunta il fatto che devono esplorare molteplici comparti ambientali quali terreno, acqua, vegetazione e aria per procacciarsi nutrimento e riparo li rende ottimi indicatori biologici. Essi sono in grado di percepire i fattori di alterazione ambientale, sia che si manifestino tramite un singolo fenomeno che per mezzo di un insieme di fattori ecologici, tale capacità consente di ricavare informazioni utili ed omogenee per definire le pressioni in atto \citep{atta}.
I prodotti dell’alveare e il corpo stesso delle api possono assimilare differenti tipologie di inquinanti e principi attivi relativi ai prodotti fitosanitari presenti nei terreni dove vivono e bottinano risorse. Le analisi di laboratorio, oltre a fornire dati sulla qualità ambientale, possono associare le cause di morte alle pratiche colturali eseguite e indicare quali specie vegetali siano state soggette a trattamento o ad un eventuale contatto accidentale con sostanze chimiche.
Inoltre, gli apoidei selvatici aventi una ligula corta spesso instaurano relazioni alimentari con differenti specie di piante e presentano una buona adattabilità alle condizioni ambientali. Mentre le api selvatiche a ligula lunga sono maggiormente dipendenti da un determinato tipo di flora e più esigenti in termini di habitat \citep{bianco}. A seconda della tipologia di pronubi presenti si possono trarre informazioni sulla situazione ecologica e sulla qualità del paesaggio, poiché gli apoidei sono legati all’ambiente in cui vivono per l’intero ciclo vitale, e la scomparsa di specie maggiormente sensibile fornisce indicazioni sulla situazione ecologica e sulle azioni antropiche. Nonostante le evidenze dimostrino un progressivo calo degli impollinatori selvatici, molteplici lacune non consentono di definire metodi univoci che sintetizzino a pieno la portata del problema misurando i cambiamenti che le popolazioni di api selvatiche subiscono negli ecosistemi agrari. Dunque, è necessario incentivare il monitoraggio a lungo termine al fine comprendere meglio i rischi legati alla gestione dell’agroecosistema e in particolare all’uso di pesticidi e prodotti di sintesi sugli impollinatori, in maniera tale da garantirne la tutela tramite l’adozione di specifiche misure.

\clearpage

\end{document}