\documentclass[main.tex]{subfiles}

\begin{document}

\subsection{Il servizio di impollinazione nelle aree agricole}

L'impollinazione delle piante da fiore da parte degli animali rappresenta un servizio ecosistemico di grande valore per l'umanità dal momento che gli impollinatori, grazie alla loro azione, sono indispensabili per la riproduzione di gran parte delle principali colture agrarie (media globale 75\%) e della quasi totalità delle Angiosperme selvatiche (media globale 90\%) \citep{bianco}.
L’impollinazione rappresenta un servizio ecosistemico essenziale e, nonostante siano state effettuate stime, il reale valore di tale azione risulta difficilmente calcolabile. Tuttavia, si valuta che in Europa circa l’84\% delle specie coltivate e il 78\% della flora selvatica dipendano dall’impollinazione animale e una cifra pari a quasi 15 miliardi di euro della produzione agricola annuale nell’UE sia direttamente attribuibile agli insetti impollinatori \citep{comeu}.
La dipendenza umana da queste piante per la fornitura di cibo, foraggio per il bestiame, medicinali e materiali di origine vegetale è notevole, tanto che il valore complessivo fornito dall’impollinazione per la produzione alimentare è stimato a circa 351 miliardi di dollari. Secondo il Terzo Rapporto sullo Stato del Capitale Naturale in Italia \citep{tr} la valutazione economica del servizio di impollinazione delle aree agricole italiane è pari a circa 2 miliardi di euro l’anno \citep{isp}.
La diversità e abbondanza delle api e il relativo servizio ecosistemico fornito dall’impollinazione, sono positivamente influenzati dall’elevata quantità e qualità delle risorse fiorali, dalla maggiore eterogeneità del paesaggio e dalla percentuale di aree naturali e semi-naturali nei paesaggi agricoli \citep{bianco}. La relazione tra impollinatori e piante coltivate rappresenta un valore economico e spesso è definita da un’elevata specializzazione tra le parti.
Gli impollinatori selvatici svolgano un ruolo vitale nell'impollinazione delle coltivazioni ma, come già sottolineato nel capitolo \ref{Cap. 1.6}, sono seriamente minacciati dall’intensificazione delle pratiche agricole e dalla perdita di biodiversità degli agroecosistemi. Per sopperire a tale perdita spesso l’impollinazione delle colture viene affidato all’ape da miele, poiché la scarsa specializzazione verso una coltura specifica abbinata al facile allevamento e trasporto in arnie artificiali la rende un impollinatore commercialmente valido.
Tuttavia, è necessario incentivare l’azione dei pronubi selvatici tutelando il loro habitat di vita tramite la conservazione dei filari, delle siepi, delle fasce inerbite, delle pozze d’acqua e dei prati situati ai margini delle colture agrarie. Tali pratiche non solo sono in grado di aumentare la diversità e l’abbondanza degli apoidei selvatici ma preservano anche i nemici naturali di parassiti e patogeni che infestano le piante coltivate, contenendo perdite di produzione ed evitando costi economici per i relativi trattamenti \citep{ken}.
In particolare, le monocolture, caratteristiche dell’agricoltura intensiva, rappresentano ambienti scarsamente favorevoli alla sopravvivenza delle api selvatiche, per la presenza di una sola tipologia di polline, spesso di scarsa qualità, e per un periodo limitato della stagione.
Quindi le api selvatiche, aventi un periodo di volo diverso da quello della coltura dominante, non sono in grado di sopravvivere in aree dove la disponibilità di polline è limitata a pochi mesi l’anno. Le colture agrarie traggono beneficio da una maggiore biodiversità di api, piuttosto che dalla presenza di un’unica specie, dal momento che un elevato numero di specie di insetti pronubi presentano differenti periodi di volo, sia durante il periodo di fioritura che all’interno della stessa giornata. Tuttavia, anche le specie di api che non sono direttamente coinvolte nel processo di impollinazione delle colture hanno un ruolo indiretto, poiché consentono la riproduzione delle piante selvatiche che fungono da risorsa alimentare agli insetti pronubi quando le colture agrarie non sono più in fiore \citep{borto}.
Per le piante coltivate una impollinazione efficace implica frutti e semi di maggiori dimensioni e più numerosi, con forme più regolari e migliori qualità organolettiche che conferiscono al prodotto finale un superiore valore commerciale. È noto che la presenza di insetti impollinatori induca per le specie da seme un aumento della quantità di prodotti e per le specie ortofrutticole un incremento della qualità del prodotto.
Dunque, è opportuno diversificare la strategia produttiva alimentare su molteplici impollinatori al fine di tutelare le produzioni vegetali e, di conseguenza, aumentare in grado di biodiversità ecosistemico.

\end{document}