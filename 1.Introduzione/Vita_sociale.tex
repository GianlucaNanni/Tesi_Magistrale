\documentclass[main.tex]{subfiles}

\begin{document}

\subsubsection{Vita sociale}

“La vita sociale è l'emergenza dell'attività di un certo numero di individui che hanno tra loro relazioni privilegiate, affinità particolari, con una modalità di interazione basata su un insieme di segnali di comunicazione reciproca, che permettono loro di svolgere ruoli particolari all’interno dei gruppi \citep{camp}.”
Dunque, per poter parlare di socialità è necessaria un’organizzazione intraspecifica che sia fondata sulla cooperazione e che vada oltre la sola componente riproduttiva.
Nel corso della storia evolutiva delle api si è più volte verificata la comparsa di comportamenti sociali, così che ora esistono specie che esibiscono pressoché tutti i livelli di socialità noti per gli insetti. Ci sono generi che presentano gradi di socialità molto complessi, dove gli individui si aggregano in colonie costituite da una sola femmina fertile, la regina, e una casta sterile costituita dalle operaie, che collabora per la ricerca di cibo e l’allevamento della prole. Tali aggregazioni comprendono anche una sovrapposizione temporale di più generazioni \citep{turil}.
Tuttavia, la maggior parte delle specie di api è solitaria: dunque, ogni femmina depone le uova e rifornisce le proprie larve di polline e nettare in maniera autonoma. Eppure, l’apparente “assenza di vita sociale” presenta varie sfaccettature, poiché molte specie solitarie sono anche “gregarie”, ovvero gli individui femminili nidificano gli uni vicino agli altri, mentre altre sono “comunitarie”, cioè condividono lo stesso nido senza che avvenga cooperazione tra individui. In alcune famiglie, come gli alittidi (Halictidae), vi sono sia specie solitarie che specie che presentano una qualche forma di pre-socialità, che può consistere nella semplice cooperazione tra femmine fertili nella costruzione del nido, fino a una forma di socialità più marcata, in cui una sola femmina si occupa della deposizione delle uova, mentre le altre sono dedite alla raccolta di cibo e alla difesa del nido \citep{aldi}.

\end{document}