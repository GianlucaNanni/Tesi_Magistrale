\documentclass[main.tex]{subfiles}

\begin{document}

\subsection{Importanza delle collezioni naturalistiche}

Le collezioni naturalistiche rappresentano un patrimonio di immenso valore non solo per la ricerca di base e applicata, ma anche per le problematiche di conservazione e tutela degli habitat naturali e per la divulgazione del sapere scientifico \href{https://www.isprambiente.gov.it/it/attivita/museo/collezioni-naturalistiche-biologiche}{(Collezioni Naturalistiche Biologiche, 2022)}.
In Italia esistono molteplici collezioni appartenenti principalmente alle Università, ne è un esempio la sede universitaria di Bologna che vanta importanti collezioni zoologiche, botaniche, paleontologiche e mineralogiche. Gli esemplari custoditi in queste collezioni rappresentano la base scientifica materiale dell’odierna conoscenza floristica e faunistica. Nello specifico le collezioni di interesse in questa tesi sono le raccolte botaniche ed entomologiche.

\end{document}