\documentclass[main.tex]{subfiles}

\begin{document}

\subsection{Ricompense fiorali: polline e nettare}\label{Cap. 1.7}

Le piante entomofile sono dotate di attraenti primari, quali polline e nettare, aventi lo scopo di attrarre insetti impollinatori: in questo modo il vettore zoofilo durante la raccolta di polline o nettare si carica di polline che, nella successiva visita, può depositare sulla parte femminile del fiore. Il polline viene sviluppato all’interno delle antere che all’opportuno momento di maturazione si aprono permettendone la dispersione. Siccome all’interno del polline sono contenuti i gameti maschili, una volta che raggiunge la parte femminile del fiore, ovvero lo stigma, può avvenire la fecondazione degli ovuli che sono contenuti nell’ovario \citep{alt}. Inoltre, il polline raccolto dalle api viene anche utilizzato come nutrimento per larve e giovani api, che consente sia il completamento dello sviluppo corporeo, sia lo sviluppo delle funzionalità degli organi quali: le ovaie, le zone adipose e le ghiandole ipofaringee.
Il polline per le api è un alimento completo perché contiene sia proteine solubili che insolubili, lipidi e diversi tipi di carboidrati (glucosio, amido, fruttosio e zuccheri più complessi). Solitamente le api prediligono la raccolta di polline privo di amido e ricco di zuccheri più semplici, perché riescono ad assimilarlo in maniera più semplice e veloce. Per giunta, grazie alla sua bassa concentrazione di acqua rappresenta la principale fonte di composti azotati, divenendo di fatto un alimento imprescindibile.
Il polline della flora entomofila è ricoperto da una sostanza di origine lipidica chiamata pollenkitt \citep{bellu}, alla quale sono attribuite molteplici funzioni: conferisce colore e odore al polline, essendo viscoso favorisce alla formazione degli agglomerati che le api trasportano nelle cestelle delle zampe posteriori e ne consente l’adesione allo stigma e all’antera finché questi non sono attivamente raccolti dall’impollinatore.
Invece, il nettare rappresenta la principale fonte di carboidrati per gli individui adulti e viene prodotto da tessuti vegetali specializzati detti nettàri. Esistono varie tipologie di nettàri e tali strutture assumono un rilevante significato ecologico dal momento che sono il luogo dove avviene la produzione di sostanze coinvolte nelle interazioni con gli animali.
Le tipologie di nettàri si distinguono in base alla collocazione che hanno sulla pianta. I nettàri fiorali sono ubicati sul fiore e svolgono la funzione di ricompensa alimentare dal momento che il nettare che contengono viene bottinato dai pronubi, mentre nel caso dei nettàri extrafiorali, che trovano la loro collocazione nella parte vegetativa della pianta, sono solitamente dediti alla difesa indiretta contro gli insetti erbivori e soltanto sporadicamente tale nettare viene bottinato e di conseguenza coinvolto nel processo di impollinazione \citep{bellu}.
Il nettare contiene prevalentemente sostanze che derivano dalla fotosintesi e la sua quantità, qualità e durata dipendono fortemente da fattori morfologici e fisiologici della pianta, dalle caratteristiche dell’habitat e dal tipo di animale impollinatore. È principalmente composto da zuccheri quali glucosio, fruttosio e saccarosio, ma in quantità minori possono essere presenti anche lipidi, proteine, amminoacidi e vari metaboliti secondari.

\end{document}