\documentclass[main.tex]{subfiles}

\begin{document}

\subsection{Digitalizzazione e interconnessioni delle collezioni}

La digitalizzazione delle collezioni è uno strumento informatico fondamentale che consente lo studio di collezioni storiche e moderne tramite sistemi tecnologici innovativi e ne garantisce la visibilità sul panorama nazionale ed internazionale.
Gli attuali sistemi catalografici informatizzati per i beni naturalistici si devono all’entrata in vigore del Codice dei beni culturali e del paesaggio (D. lgs.42/2004 e s.m., art. 10) che ha interessato i reperti naturalistici aventi valore storico e conservati nei musei pubblici, elevandoli a beni culturali e permettendone la specifica tutela a tutti gli effetti \citep{corra}.
I progetti di digitalizzazione tramite le moderne tecnologie informatiche consentono un facile accesso al patrimonio culturale e scientifico e, oltre ad esaltarne le potenzialità, ne hanno definito nuovi metodi di diffusione dei contenuti e di interconnessione dei patrimoni.
L’implementazione delle tecnologie informatiche consente una sempre maggiore fruibilità dei reperti ed al contempo ne tutela l’integrità, evitando sia lo spostamento fisico dello studioso che intende analizzarlo o, al contrario, lo spostamento del campione stesso con tutti i rischi che ne conseguono, come deterioramento o perdita dello stesso. La smaterializzazione degli oggetti reali, attraverso la diffusione tramite la rete digitale, consente di mantenere la divulgazione della conoscenza e la valorizzazione separate dai problemi di conservazione.
Inoltre, l’informatizzazione delle collezioni di erbario e delle relative informazioni consente la condivisione e l‘aggregazione di dati e metadati bio-tassonomici creando un eccellente sistema di ricerca collaborativo. La disponibilità di grandi quantità di dati, prodotti da studiosi e ricercatori da ogni parte del mondo, ha consentito indagini impensabili prima dell’era di Internet, comportando profonde implicazioni nell’organizzazione della produzione scientifica \citep{def}.
Nella presente tesi è di specifico interesse sottolineare l’importanza della digitalizzazione nelle collezioni d’erbario, poiché costituiscono uno strumento fondamentale per la ricerca botanica. Per consentire la gestione e la messa online dei campioni d’erbario è necessario uno database specifico, il quale rende possibile la catalogazione dei metadati e il collegamento alla relativa scansione ad alta risoluzione.
La prima fase della digitalizzazione consiste nell’imaging del campione. Tale procedura avviene per mezzo di uno scanner a letto piatto capovolto che solleva il campione fino alla superficie di scansione, in modo tale da non capovolgere mai un campione di erbario, poiché tale pratica risulterebbe dannosa alla conservazione dei campioni stessi. Successivamente avviene l’associazione dell’immagine al relativo schedario dei metadati \citep{mini}. Le informazioni che ogni singolo campione porta con sé, allegate sul cartellino, sono definite come metadati e durante la digitalizzazione vengono contrassegnate da un univoco codice di riconoscimento affinché non avvengano inesattezze.
I maggiori database gestionali per erbari virtuali richiedono alcuni campi di classificazione dei singoli campioni riconducibili alle informazioni di campo e di catalogazione, quali: tag e prefisso di identificazione, categoria della collezione, data e autore della raccolta, stato del campione, famiglia di appartenenza, nome scientifico e nome comune, nominativo della località e relative note ambientali, stock di appartenenza iniziale e attuale, coordinate latitudinali, longitudinali ed altimetriche.
Gli erbari digitali consentono dunque un’ampia diffusione dei loro contenuti e rendono possibile a ricercatori e studiosi di relazionare campioni d'erbario, dati biologici e nomenclatura botanica, oltre a condividere questo importante patrimonio collettivo con il grande pubblico ed avvicinare le persone alle tematiche ambientali ed ecologiche \citep{cleme}.

\end{document}