\documentclass[main.tex]{subfiles}

\begin{document}

\section{INTRODUZIONE}
\subsection{Funzione dell’Erbario}

L’erbario è una raccolta di piante conservate, catalogate e sistemate secondo un ordine sistematico per lo studio e la consultazione. I campioni che costituiscono le collezioni di erbario oltre a rappresentare un patrimonio genetico in termini di biodiversità costituiscono una inestimabile fonte di conoscenza \href{https://www.kew.org/science/collections-and-resources/collections/herbarium}{(“The Herbarium”, 2019)}.
Il termine erbario assume un doppio significato: esso può identificare sia una raccolta di piante essiccate, che una struttura museale espressamente dedicata alla conservazione e alla consultazione di tale materiale.
L'Erbario custodisce campioni di piante essiccate che vengono preservati e curati nel tempo, in maniera tale che le generazioni future possano utilizzare la flora in oggetto per scopi identificativi \citep{blunt}, di studio della biodiversità e per uno sviluppo sostenibile.
Un esemplare di erbario può essere costituito da una pianta intera, nel caso si tratti di erbe di piccole dimensioni, o da parti di una pianta qualora siano grandi alberi o cespugli che ne impediscono la raccolta per intero. I campioni solitamente includono radici, fusto, foglie, corteccia e preferibilmente dovrebbero essere comprensivi di fiori e/o frutti, e in generale di tutti gli elementi diagnostici. Il materiale raccolto varia a seconda della specie, e potrebbe includere anche elementi corposi come parti legnose, bulbi e frutti secchi.
Per identificare in modo corretto le specie osservate o raccolte in campo e determinarne le differenze morfologiche può essere molto utile il confronto con campioni di exsiccata già catalogati.
I campioni di erbario hanno la funzione di fonte di informazioni \citep{raf}: sono utili per indagini morfometriche e tassonomiche, per merito delle informazioni associate alla loro raccolta è possibile integrare le conoscenze biogeografiche, ecologiche e fenologiche nel caso di specie a rischio di estinzione costituiscono una insostituibile fonte di dati storici. Inoltre, possono assolvere la funzione di convalidare le osservazioni scientifiche e se necessario fornire materiale genetico (DNA) per studiare relazioni e processi evolutivi.
Le moderne tecnologie consentono di includere tra i dati finalizzati all’identificazione la geolocalizzazione del luogo di raccolta tramite coordinate GPS. È quindi possibile registrare latitudine e longitudine sul campo, evitando di calcolare a posteriori l’ubicazione esatta, come invece spesso accade per i campioni più datati e comprensivi di idonee informazioni registrate sull'etichetta o fonti come i dizionari geografici. È sempre importante quindi che su ogni cartellino vengano riportate informazioni necessarie come: data, località, nome delraccoglitore e dell’identificatore se diversi, nome della specie e della famiglia, e altre note utili come la descrizione della popolazione o della località di raccolta.
Nel caso dei moderni erbari i campioni vengono organizzati sistematicamente, ovvero per famiglia, genere e specie, al fine di avvicinare gli elementi simili e facilitarne il confronto. Non sempre tale organizzazione è applicabile poiché opere storiche raccolte sotto forma di volumi rilegati non consentono questo genere di disposizione senza deturparne il valore storico.
L’erbario assume dunque molteplici funzioni, modificatesi nel tempo e tuttora in costante evoluzione, dove le specie vegetali sono raccolte e curate per scopi di ricerca, conservazione e divulgazione del sapere. Attualmente espletano funzioni riconducibili all’interesse storico, scientifico ed educativo.
In particolare, l’ultima evoluzione degli Erbari è costituita dalla digitalizzazione dei campioni stessi, che è divenuto un metodo di enorme utilità per conservare e divulgare informazioni sulla biodiversità vegetale e sulle collezioni storiche. La digitalizzazione è un eccellente sistema per assicurare da un lato la sempre maggiore fruibilità dei reperti e dall’altro una migliore conservazione degli stessi. In questa maniera viene attuata una metodologia non invasiva che da un lato rassicura le istituzioni che detengono tali patrimoni su eventuali danni fisici e dall’altro dà la possibilità di consultare i materiali digitali da remoto ottimizzando le tempistiche di fruizione \citep{rad}.

\end{document}