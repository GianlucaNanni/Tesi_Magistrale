\documentclass[main.tex]{subfiles}

\begin{document}

\subsection{Importanza degli impollinatori, con particolare riferimento alle api selvatiche}\label{Cap. 1.6}

Il rapporto delle api con la flora è di primaria importanza per la funzione di impollinazione delle colture, attività dalla quale dipende fortemente la stessa sopravvivenza del genere umano.
Gli insetti pronubi e le piante entomofile sono legati da un rapporto di mutua dipendenza, dove gli insetti dipendono dalle piante come fonti alimentari, mentre le piante dipendono dagli insetti per la riproduzione. Per ovviare alla problematica del deficit di impollinazione delle colture agrarie, sono spesso introdotte api mellifere per il "servizio" di impollinazione, capaci di visitare molte piante ma in modo non specifico, a scapito delle autoctone api specialiste, che visitano in modo preferenziale i fiori di una determinata specie vegetale, della quale sono impollinatori molto efficienti \citep{gomez}.
Questo ha portato all’allevamento di apoidei per scopi applicativi, quali \textit{Apis mellifera} per il servizio di impollinazione soprattutto di fruttifere e sementiere, oltre la produzione di miele, e alcune specie di bombi, come \textit{Bombus terrestris} \href{https://www.apicoltura.ch/apidologia/i-bombi.html}{(I bombi, 2022)}, per l’impollinazione di colture in serra come il pomodoro. Il ricorso a specie allevate, soprattutto se aliene, può risultare rischioso per le api selvatiche autoctone a causa della competizione per le risorse disponibili e per il potenziale inquinamento del corredo genetico, con conseguente perdita di biodiversità. È dunque auspicabile cercare di trarre il massimo profitto dal servizio di impollinazione fornito dalle popolazioni naturali di api selvatiche \citep{yoko}.
Le evidenze scientifiche mostrano un netto declino degli impollinatori selvatici, gli studi svolti hanno evidenziato che si tratta di un fenomeno su scala planetaria, dovuto all’effetto combinato di una pluralità di cause. Le principali minacce alla sopravvivenza delle api selvatiche sono riconducibili a: cambiamenti climatici in atto a livello globale, che comportano un progressivo innalzamento delle temperature su gran parte delle terre emerse con conseguente modificazione delle cenosi vegetali; la trasformazione del paesaggio dovuta all’urbanizzazione, causa di riduzione e frammentazione degli habitat adatti alla sopravvivenza delle api selvatiche \citep{lebuh}. Le maggiori criticità sono riscontrate nell’intensificazione delle pratiche agricole, che comporta l’eliminazione della flora spontanea, l’estensione delle monocolture, la compattazione e il degrado dei suoli (dove nidifica la maggior parte delle specie). Il deflusso di azoto causa eutrofizzazione e l’inquinamento delle acque e l’imponente impiego di prodotti fitosanitari, quali insetticidi, fungicidi ed erbicidi sintetici rende gli impollinatori più vulnerabili agli altri fattori di pressione.
È doveroso sottolineare che la perdita di impollinatori interrompe una catena alimentare che condiziona pesantemente altre specie viventi e l’intera rete ecologica, oltre a mettere a rischio la sicurezza alimentare umana. La tutela degli impollinatori consente la conservazione della biodiversità vegetale e degli ecosistemi naturali, limitando eventuali azioni di ripristino antropiche.

\end{document}