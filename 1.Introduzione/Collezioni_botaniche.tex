\documentclass[main.tex]{subfiles}

\begin{document}

\subsubsection{Collezioni botaniche}

Le collezioni botaniche, costituite da esemplari vegetali essiccati, rappresentano un patrimonio biologico di primaria importanza e assumono molteplici e rilevanti funzioni, principalmente imputabili alla ricerca scientifica. Tuttavia, la valenza di un erbario non è da considerarsi limitatamente all’ambito botanico-naturalistico ma sottintende un’ulteriore rilevanza culturale, sociale e didattica.
In particolare, le collezioni botaniche costituite dagli erbari storici testimoniano il progredire della scienza botanica e possono contribuire a tracciarne la storia, dal momento che la conservazione di piante essiccate è una pratica consolidata ormai da diversi secoli \citep{moggi}. Ne sono un esempio i cartellini dei campioni d’erbario, specialmente quelli più antichi, se opportunamente compilati raccontano parte della storia dei naturalisti che li hanno raccolti e preparati divenendo di conseguenza strumenti utili per ricostruzioni storico-biografiche \citep{bonin}. Oltre a ciò, sono utili per sottolineare l’evoluzione avvenuta in ambito tassonomico, qualora una specie dovesse migrare o essere suddivisa in differenti categorie sistematiche, e fornire prove relative alla passata denominazione scientifica. Dunque, il materiale più datato accompagnato dai corrispettivi cartellini, presenta un valore più storico che scientifico.
L’importanza principale degli esemplari d’erbario si configura in special modo nel loro valore scientifico, che si palesa sotto vari aspetti. Un campione d’erbario debitamente munito di cartellino che testimonia la data ed il luogo in cui è stato raccolto risulta estremamente informativo. Inoltre, gli exsiccata possono rappresentare una valida fonte di dati per nuovi studi scientifici oppure fornire materiale per analisi genetiche, costituendo una più accessibile fonte di materiale biologico rispetto alla ricerca di nuovo materiale in natura, il che risulta particolarmente utile quando si studiano specie particolarmente rare o effimere \citep{greve}.

\end{document}