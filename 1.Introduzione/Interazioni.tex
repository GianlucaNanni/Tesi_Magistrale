\documentclass[main.tex]{subfiles}

\begin{document}

\subsection{Interazioni piante-impollinatori}

Le api sono un gruppo monofiletico di Imenotteri Aculeati che a livello mondiale comprende quasi 20.000 specie, delle quali 2.000 si trovano in Europa e circa 1.000 sono presenti nella sola Italia \href{https://www.wwf.it/pandanews/ambiente/impollinatori-ditalia/}{(Impollinatori d’Italia, 2020)}. Dal punto di vista tassonomico le api sono solitamente ascritte al clade Antophila, incluso nella superfamiglia Apoidea insieme alle vespe sfecoidi. Quando si pensa alle api solitamente si fa riferimento ad \textit{Apis mellifera} Linnaeus, 1758, la comune ape occidentale da miele, trascurando l’esistenza di un elevato numero di specie selvatiche che per merito delle proprie caratteristiche morfologiche, quali corpo rivestito fittamente da setole e l’apparato boccale succhiante o lambente-succhiante, sono efficienti impollinatori e garantiscono la riproduzione di molte specie vegetali \citep{dipri}.
La totale dipendenza delle api dai fiori comporta che, sia gli individui adulti che quelli allo stadio larvale, trovino il proprio nutrimento quasi esclusivamente nel nettare e polline. Tale dipendenza trofica con le piante a fiore ha consentito la reciproca coevoluzione tra antenati delle api e strutture fiorali, durante la quale le api hanno sviluppato comportamenti e strutture anatomiche che consentono loro di massimizzare la raccolta di nettare e polline, mentre nelle piante è avvenuta la modificazione delle strutture fiorali con conseguente disponibilità delle ricompense alimentari così da favorire l’azione pronuba delle api \citep{kk}.
Le api per raccogliere il polline sono dotate di peli piumosi disposti, più o meno abbondantemente, su tutto il corpo ed in particolar modo nelle zampe e nella regione sternale dell’addome nelle femmine. Oltre a ciò, le tribù degli Apini e dei Bombini per il trasporto del polline presentano strutture specializzate nelle zampe posteriori, dette corbicule \citep{manipo}, che costituiscono una caratteristica distintiva poiché consentono un ulteriore accumulo di polline. Il processo evolutivo ha comportato nell’apparato boccale delle api delle progressive modificazioni per consentire la manipolazione di tali risorse alimentari quali il lambire e succhiare liquidi zuccherini. Bisogna specificare che il rapporto che si è stabilito tra api e flora è altamente dipendente da alcune caratteristiche morfologiche dell’ape in rapporto alla forma e dimensione del fiore: la lunghezza della ligula, la robustezza e la taglia dell’insetto e l’ubicazione delle relative strutture per la raccolta del polline \citep{borto}. L’apparato boccale con cui le api suggono il nettare dai fiori è definito ligula. Se ne possono distinguere varie tipologie in base alla lunghezza, poiché ligule corte sono capaci di raggiungere soltanto fiori con calice superficiale e nettàri esterni (es. Asteraceae e Apiaceae) mentre ligule lunghe possono visitare fiori con calice allungato e nettàri profondi (es. Ericaceae) \citep{filis}.
L’attività fondamentale che svolgono le api durante il processo di alimentazione, e che apporta notevoli benefici per l’essere umano, è l’impollinazione. Gli imenotteri apoidei (superfamiglia Apoidea), oltre a comprendere sia specie selvatiche che domestiche, sono di fatto gli insetti più numerosi fra tutti i pronubi. Tramite l’azione di trasporto del polline da un fiore ad un altro consentono l’impollinazione, senza la quale non si avrebbe la fecondazione con l’unione dei due gameti (maschile e femmine) \citep{form}. Di fatto rendono possibile la riproduzione incrociata delle specie vegetali e la diversificazione del pool genico nella popolazione.
L’assenza di impollinatori limiterebbe fortemente, se non del tutto, la riproduzione di molte specie vegetali e viceversa l’assenza di una flora diversificata causerebbe la mancanza di risorse alimentari per molti insetti impollinatori, implicando effetti a cascata sull’intera catena trofica. Dunque, questa interazione mutualistica consente un efficace strategia sia per le piante in termini riproduttivi che per gli apoidei impollinatori in termini di nutrimento, poiché ne traggono polline (principalmente per il nutrimento delle larve) e nettare (per il nutrimento degli adulti) \citep{bellu}.
In particolare, le popolazioni di api selvatiche, visitando fiori di individui diversi, consentono la riproduzione incrociata di molte specie e varietà autosterili, ed è stato dimostrato che l’impollinazione incrociata determina spesso una migliore qualità dei frutti. Considerando anche le piante spontanee, l’autoimpollinazione di alcune specie viene quindi limitata attenuando di conseguenza la deriva genetica da depressione di consanguineità. Per alcune colture agricole, per cui le api mellifere sono impollinatori meno efficienti rispetto alle api selvatiche, queste assumono anche un valore economico \citep{capo}.
In ambito agronomico, il ruolo che le api selvatiche svolgono con l’impollinazione è da considerarsi complementare, e non in concorrenza, con quello dell'ape mellifera. La variabilità delle forme e dimensioni permette loro di raggiungere fiori che sono “inutilizzabili” dall'ape domestica; quindi, sono da considerarsi una risorsa fondamentale per l’impollinazione di piante coltivate ad uso alimentare.

\end{document}