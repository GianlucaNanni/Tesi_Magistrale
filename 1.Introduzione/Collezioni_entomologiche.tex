\documentclass[main.tex]{subfiles}

\begin{document}

\subsubsection{Collezioni entomologiche}

Le collezioni entomologiche sono spesso costituite da insetti dalle provenienze più varie e possono comprendere enormi quantità di esemplari. I campioni entomologici sono facilmente soggetti a deterioramento a causa della fragilità intrinseca e all’azione di muffe e parassiti, che possono portare conseguenze irreparabili in pochissimo tempo qualora non si agisca tempestivamente \citep{barto}. Inoltre, per assicurare la corretta preservazione dei campioni è opportuno, a seguito della disidratazione e messa in posizione degli stessi, che siano conservati in scatole preferibilmente di legno aventi coperchio superiore in vetro così da garantirne la visione ed al contempo assicurare una idonea chiusura ermetica.
Le collezioni storiche spesso hanno un valore che eccede il puro interesse tassonomico, perché di fatto sono la testimonianza di un popolamento faunistico passato, che potrebbe non esistere più, verosimilmente a seguito di cambiamenti climatici e ambientali, principalmente causati dall’azione antropica. Dunque, le uniche testimonianze permangono nelle suddette collezioni \citep{mone}.
Questi dati assumono quindi un valore maggiore perché permettono di valutare l’influenza dei cambiamenti ambientali sulla distribuzione geografica e temporale delle specie, consentendo di formulare utili modelli previsionali futuri, possibilità che di fatto sarebbe impossibile senza le banche-dati delle collezioni. Oltre a ciò, come già accennato anche per le collezioni botaniche, la presenza di campioni di riferimento (typus) consente la risoluzione di vari problemi tassonomici e sistematici conferendo alla collezione un elevato valore scientifico.
La necessità di raccogliere campioni, animali o vegetali, è correlata all’identificazione tassonomica degli stessi. In alcuni casi l’osservazione delle caratteristiche morfologiche può avvenire soltanto in laboratorio tramite l’ausilio dello stereomicroscopio poiché i caratteri diagnosti sono spesso microscopici.
Considerata la solida valenza ed ampiezza delle collezioni entomologiche, per le tematiche oggetto di questa analisi sono di particolare interesse le informazioni sulle interazioni pianta(fiore)-insetto impollinatore poiché forniscono preziose indicazioni utili a fini conservazionistici.

\end{document}