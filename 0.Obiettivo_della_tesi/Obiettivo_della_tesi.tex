\documentclass[main.tex]{subfiles}

\begin{document}

\title{Obiettivo della tesi}

\begin{Large}\textbf{Obiettivo della tesi}\end{Large}
\vspace{5mm}

L’obiettivo del presente elaborato è di analizzare le interazioni della rete piante-impollinatori presenti negli agroecosistemi intensivi e seminaturali del Progetto di Monitoraggio Nazionale BeeNet e conseguentemente implementare la collezione di piante essiccate dell’Erbario dell'Università di Bologna con nuovi campioni. \\
L’interesse scientifico su cui si basa questo elaborato si origina dalla necessità di comprendere e tutelare le relazioni mutualistiche tra flora entomofila ed insetti pronubi, con particolare riferimento alle api selvatiche che popolano gli agroecosistemi. \\
Lo studio effettuato in questo lavoro di tesi parte da queste premesse. Nella prima parte del presente elaborato sono definite le valenze delle collezioni naturalistiche, con particolare enfasi sull’utilità degli erbari, e sono introdotti i principi dell’impollinazione in riferimento agli apoidei selvatici in aree agricole. La seconda frazione assume uno scopo più applicativo. I campionamenti di flora entomofila e pronubi effettuati nei siti BeeNet fungono da materiale di partenza per definire i metodi di indagine, finalizzati alla comprensione dei dati botanici e del network tra piante e impollinatori.

\setcounter{page}{1}
\clearpage

\end{document}