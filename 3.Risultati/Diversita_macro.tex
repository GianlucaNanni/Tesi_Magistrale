\documentclass[main.tex]{subfiles}

\begin{document}

\subsection{Diversità floristica a confronto tra regioni del Nord/Centro/Sud/Isole Italia}

Nella presente analisi l’indice di similarità di Sørensen è stato calcolato considerando il rango di genere, per il confronto qualitativo degli agroecosistemi intensivi e seminaturali delle macroregioni, definite nel Cap. \ref{Cap. 2.9}. \\
La macroregione definita NORD è comprensiva dei siti intensivi VEAI, FRAI, ERAI, ERAI e seminaturali VEES, FRES, ERESP, PIES e l’indice di similarità di Sørensen ottenuto da tale comparazione è:

\centering
0.3548387
\medskip

\raggedright
La macroregione definita CENTRO è comprensiva dei siti intensivi TOAI, UMAI e seminaturali TOES, UMES e l’indice di similarità di Sørensen ottenuto da tale comparazione è:

\centering
0.5223881
\medskip

\raggedright
La macroregione definita SUD è comprensiva dei siti intensivi ABAI, CAAI, PUAI e seminaturali ABES, CAESD, PUES e l’indice di similarità di Sørensen ottenuto da tale comparazione è:

\centering
0.3508772
\medskip

\raggedright
La macroregione definita ISOLE è comprensiva dei siti intensivi SAAI, SIAI e seminaturali SAES, SIES e l’indice di similarità di Sørensen ottenuto da tale comparazione è:

\centering
0.3448276

\end{document}





