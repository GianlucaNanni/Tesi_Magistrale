\documentclass[main.tex]{subfiles}

\begin{document}

\section{RISULTATI}
\subsection{La flora entomofila rilevata nei siti di studio}

Il monitoraggio della flora entomofila effettuato dal Progetto BeeNet ha determinato la seguente ripartizione numerica per specie e categorie tassonomiche di rango superiore rilevate nei relativi siti di studio (Tab. \ref{tab:5}). Tale indagine ha evidenziato il numero totale dei campioni per ognuna delle 11 regioni di interesse, per tipologia di agroecosistema (intensivo o semi-naturale) regionale, definendo il numero di famiglie e generi presenti per ogni sito:

\begin{table}[!ht]
    \centering
    \begin{tabular}{|p{2,2cm}|p{2.2cm}|p{1.3cm}|p{1.5cm}|p{1.6cm}|p{1.6cm}|}
    \hline
        \textbf{Regioni} & \textbf{Totale delle specie per regione} & \textbf{Siti BeeNet} & \textbf{Specie per sito} & \textbf{Generi per sito} & \textbf{Famiglie per sito} \\ \hline    
        Abruzzo & 148 & ABAI & 75 & 65 & 23 \\ \hline
        ~ & ~ & ABES & 73 & 65 & 24 \\ \hline
        Campania & 133 & CAAI & 58 & 54 & 23 \\ \hline
        ~ & ~ & CAESD & 75 & 66 & 29 \\ \hline
        E. Romagna & 91 & ERAI & 36 & 34 & 20 \\ \hline
        ~ & ~ & ERESP & 55 & 52 & 21 \\ \hline
        Friuli V.G. & 107 & FRAI & 49 & 44 & 20 \\ \hline
        ~ & ~ & FRES & 58 & 49 & 22 \\ \hline
        Piemonte & 94 & PIAI & 36 & 33 & 17 \\ \hline
        ~ & ~ & PIES & 58 & 50 & 22 \\ \hline
        Puglia & 90 & PUAI & 27 & 27 & 16 \\ \hline
        ~ & ~ & PUES & 63 & 54 & 24 \\ \hline
        Sardegna & 189 & SAAS & 90 & 79 & 30 \\ \hline
        ~ & ~ & SAES & 99 & 69 & 33 \\ \hline
        Sicilia & 68 & SIAI & 28 & 30 & 15 \\ \hline
        ~ & ~ & SIES & 38 & 38 & 17 \\ \hline
        Toscana & 55 & TOAI & 28 & 24 & 17 \\ \hline
        ~ & ~ & TOES & 27 & 24 & 11 \\ \hline
        Umbria & 112 & UMAI & 42 & 38 & 19 \\ \hline
        ~ & ~ & UMES & 70 & 64 & 24 \\ \hline
        Veneto & 77 & VEAI & 24 & 25 & 17 \\ \hline
        ~ & ~ & VEES & 53 & 42 & 16 \\ \hline
        Totale & 1159 & 22 & 1159 & 1026 & 460 \\ \hline
    \end{tabular}
        \caption{numero di taxa per regioni e siti di studio.}
    \label{tab:5}
\end{table}

\end{document}





