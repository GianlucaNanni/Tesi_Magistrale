\documentclass[main.tex]{subfiles}

\begin{document}

\subsection{Diversità floristica a confronto tra AI (agroecosistema intensivo) ed ES (agroecosistema seminaturale)}

L’indice di similarità di Sørensen calcolato considerando i generi presenti nei transetti dei siti in agroecosistemi intensivi ABAI, CAAI, ERAI, FRAI, PIAI, PUAI, SAAI, SIAI, TOAI, UMAI, VEAI e seminaturali ABES, CAESD, ERESP, FRES, PIES, PUES, SAES, SIES, TOES, UMES, VEES, definito precedentemente al Cap. \ref{Cap. 2.8}, risulta pari a: 

\centering
0.2916667
\medskip

Mentre l’indice di Sørensen calcolato considerando le specie vegetali presenti nei siti intensivi (ERAI) e seminaturali (ERESP) dell’Emilia-Romagna corrisponde a:

\centering
0.5434783

\end{document}





