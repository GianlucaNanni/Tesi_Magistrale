\documentclass[main.tex]{subfiles}

\begin{document}

\subsection{Specie rare ed endemiche presenti nei siti di studio}

Le specie vegetali entomofile sono state comparate con “La Lista Rossa della Flora Italiana” redatta dal Comitato Italiano IUCN e sono state identificate, in base ai relativi criteri di selezione (Cap. \ref{Cap. 2.4}), 6 Specie rare (SR) e 1 Specie endemica (EN) (Tab. \ref{tab:6}).

\begin{table}[!ht]
    \centering
    \begin{tabular}{|l|l|l|l|l|}
    \hline
        \textbf{Famiglia} & \textbf{Genere} & \textbf{Specie} & \textbf{SR} & \textbf{EN} \\ \hline    
        Amaryllidaceae & \textit{Allium} & \textit{Allium triquetrum} L. & X & ~ \\ \hline
        Asparagaceae & \textit{Bellevalia} & \textit{Bellevalia romana} Rchb. & X & ~ \\ \hline
        Compositae & \textit{Cirsium} & \textit{Cirsium tenoreanum} Petr. & X & X \\ \hline
        " & " & \textit{Cirsium vulgare} (Savi) Ten. & X & ~ \\ \hline
        Iridaceae & \textit{Crocus} & \textit{Crocus versicolor} Ker Gawl. & X & ~ \\ \hline
        Lythraceae & \textit{Lythrum} & \textit{Lythrum hyssopifolia} L. & X & ~ \\ \hline
    \end{tabular}
        \caption{specie rare ed endemiche di interesse.}
    \label{tab:6}
\end{table}

\end{document}

\textit{Allium}



