\documentclass[main.tex]{subfiles}

\begin{document}

\subsection{Appendice 2}

\begin{table}[!ht]
    \centering
    \begin{tabular}[\footnotesize]{|l|l|p{5cm}|p{5.5cm}|}
    \hline
        \textbf{Famiglia} & \textbf{Genere} & \textbf{Specie} & \textbf{CS} \\ \hline
        Asparagaceae & Loncomelos & Loncomelos brevistylum (Wolfner) Dostál & Presente in Erbario Generale come sinonimo di Ornithogalum pyramidale L. \\ \hline
        Brassicaceae & Erucastrum  & Erucastrum incanum W.D.J.Koch & Presente in Erbario Generale come sinonimo di Hirschfeldia incana L. \\ \hline
        ~ & Microthlaspi & Microthlaspi perfoliatum (L.) F.K.Mey. & Presente in Erbario Generale come sinonimo di Thlaspi perfoliatum L. \\ \hline
        Cistaceae & Cistus & Cistus scoparius & Nome scientifico inesistente \\ \hline
        Compositae & ~ & Asteraceae* & Nome scientifico inesistente \\ \hline
        ~ & ~ & Composita gialla & Nome scientifico inesistente \\ \hline
        ~ & Hypochaeris & Hypochaeris & Nome scientifico incompleto \\ \hline
        Convolvulaceae & Convolvulus & Convolvulus sepium L. & Presente in Erbario Generale come sinonimo di Calystegia sepium L. \\ \hline
        Dipsacaceae & Sixalix & Sixalix atropurpurea (L.) Greuter \& Burdet & Sinonimo di Scabiosa atropurpurea L., della quale è già richiesta l'erborizzazione  \\ \hline
        Lamiaceae & Salvia & Salvia clandestina L. & Presente in Erbario Storico come sinonimo di Salvia verbenaca L. \\ \hline
        Leguminosae & Coronilla & Coronilla securidaca L. & Presente in Erbario Generale come sinonimo di Securigera securidaca L. Presente in Erbario Storico come sinonimo di Securigera coronilla L. \\ \hline
        ~ & Securigera & Securigera securidaca (L.) Degen \& Dörfl. & Presente in Erbario Generale come sinonimo di Coronilla securidaca.  \\ \hline
        ~ & Sulla & Sulla coronaria (L.) Medik & Nome scientifico ambiguo. Assenza di sinonimi. \\ \hline
        ~ & Trifolium & Trifolium pentaphyllum & Nome scientifico inesistente. \\ \hline
        Orobanchaceae & Bellardia & Bellardia trixago (L.) All. & Presente in Erbario Generale e Storico come Bartsia trixago L. \\ \hline
        ~ & ~ & Bellardia viscosa (L.) Fisch. \& C.A.Mey. & Presente in Erbario Generale come sinonimo di Parentucellia viscosa L. \\ \hline
        Rosaceae & Poterium & Poterium sanguisorba & Presente in Erbario Generale e Storico come Sanguisorba minor L. \\ \hline
    \end{tabular}
        \caption{singolarità relative alla nomenclatura riscontrate durante l’analisi comparativa.}
    \label{tab:11}
\end{table}

Abbrev.: CS = casi particolari.

\clearpage

\end{document}





