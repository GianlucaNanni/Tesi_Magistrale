\documentclass[main.tex]{subfiles}

\begin{document}

\section{APPENDICE}
\subsection{Appendice 1}

\begin{table}[!ht]
    \centering
    \begin{tabular}[\footnotesize]{|p{2.2cm}|p{2.2cm}|p{2.2cm}|p{1.1cm}|p{1.6cm}|p{0.8cm}|p{1cm}|p{0.8cm}|}
    \hline
        \textbf{Famiglia} & \textbf{Genere} & \textbf{Specie} & \textbf{EG-NP} & \textbf{AGR} & \textbf{RC} & \textbf{OOSG} & \textbf{SD} \\ \hline
        Amaryllidaceae & Narcissus & Narcissus var. ornamentale & ~ & Friuli (FRES) & X & ~ & X \\ \hline
        Apiaceae & Ammi & Ammi majus L. & X & Sicilia (SIAI) & X & ~ & ~ \\ \hline
        ~ & ~ & Ammi visnaga Gaertn. & X & Sicilia (SIAI/SIES) & X & ~ & ~ \\ \hline
        ~ & Oenanthe & Oenanthe sp. & ~ & Campania (CAESD) & X & ~ & X \\ \hline
        ~ & Rouya & Rouya sp. & X & Emilia-Romagna (ERESP) & X & ~ & X \\ \hline
        ~ & Scandix & Scandix sp.  & ~ & Emilia-Romagna (ERAI) & X & ~ & X \\ \hline
        ~ & Smyrnium & Smyrnium olusatrum L. & ~ & Sardegna (SAAI) & X & X & ~ \\ \hline
        ~ & Visnaga & Visnaga daucoides Gaertn. & X & Campania (CAAI) & X & ~ & ~ \\ \hline
        Apocynaceae & Periploca & Periploca graeca L. & ~ & Toscana (TOAI) & X & X & ~ \\ \hline
        Asparagaceae & Ornithogalum  & Ornithogalum sp. & ~ & Emilia-Romagna (ERAI/ERESP) & X & ~ & X \\ \hline
        ~ & ~ & Ornithogalum comosum L. & ~ & Abruzzo (ABAI/ABES) & X & X & ~ \\ \hline
        Boraginaceae & Cerinthe & Cerinthe major L. & ~ & Abruzzo (ABAI); Puglia (PUAI); Sardegna (SAAI) & X & X & ~ \\ \hline
        ~ & Cynoglossum & Cynoglossum creticum Mill. & X & Sardegna (SAAI/SAES) & X & ~ & ~ \\ \hline
            Brassicaceae & Arabidopsis & Arabidopsis sp. & X & Piemonte (PIAI) & X & ~ & ~ \\ \hline
        ~ & Brassica  & Brassica sp. & ~ & Campania (CAAI/CAESD) & X & ~ & X \\ \hline
        ~ & Cardamine  & Cardamine sp. & ~ & Campania (CAESD) & X & ~ & X \\ \hline
        ~ & Lepidium & Lepidium draba L. & ~ & Emilia-Romagna (ERAI) & X & X & ~ \\ \hline
        ~ & Sinapis  & Sinapis sp. & ~ & Campania (CAESD) & X & ~ & X \\ \hline
    \end{tabular}
    \end{table}
    
    \clearpage
        
        \begin{table}[!ht]
        \centering
    \begin{tabular}[\footnotesize]{|p{2.2cm}|p{2.2cm}|p{2.2cm}|p{1.1cm}|p{1.6cm}|p{0.8cm}|p{1cm}|p{0.8cm}|}
    \hline
        Caprifoliaceae & Cephalaria & Cephalaria transsylvanica L. & X & Abruzzo (ABAI/ABES) & X & ~ & ~ \\ \hline
        ~ & Scabiosa & Scabiosa atropurpurea L. & X & Sicilia (SIES); Toscana (TOES); Umbria (UMAI/UMES) & X & ~ & ~ \\ \hline
        Caryophyllaceae & Cerastium & Cerastium glomeratum Thuill. & X & Abruzzo (ABAI/ABES); Campania (CAAI); Friuli (FRAI); Piemonte (PIAI/PIES); Sardegna (SAAI/SAES) & X & ~ & ~ \\ \hline
        ~ & Paronychia & Paronychia sp. & ~ & Sardegna (SAAI) & X & ~ & X \\ \hline
        ~ & Petrorhagia & Petrorhagia dubia Raf. & X & Sardegna (SAES) & X & ~ & ~ \\ \hline
        Compositae & Achillea & Achillea roseoalba Ehrend. & X & Friuli (FRES) & X & ~ & ~ \\ \hline
        ~ & Ambrosia & Ambrosia artemisiifolia L. & X & Piemonte (PIES) & X & ~ & ~ \\ \hline
        ~ & ~ & Ambrosia psilostachya DC. & X & Piemonte (PIAI) & X & ~ & ~ \\ \hline
        ~ & Anacyclus & Anacyclus clavatus (Desf.) Pers. & X & Sicilia (SIAI); Toscana (TOAI) & X & ~ & ~ \\ \hline
        ~ & Anthemis & Anthemis sp. & ~ & Piemonte (PIAI/PIES) & X & ~ & X \\ \hline
        ~ & Aster & Aster subulatus Michx. & ~ & Campania (CAAI) & X & X & ~ \\ \hline
        ~ & Bellis  & Bellis sp. & ~ & Puglia (PUES) & X & X & X \\ \hline
        ~ & Calendula & Calendula sp. & ~ & Emilia-Romagna (ERAI) & X & ~ & X \\ \hline
        ~ & Carduus & Carduus sp. & ~ & Sicilia (SIAI) & X & ~ & X \\ \hline
        ~ & Carthamus & Carthamus caeruleus L. & X & Sicilia (SIES) & X & ~ & ~ \\ \hline
        ~ & Cirsium & Cirsium tenoreanum Petr. & ~ & Abruzzo (ABES) & X & X & ~ \\ \hline
    \end{tabular}
    \end{table}
    
    \clearpage
        
        \begin{table}[!ht]
        \centering
    \begin{tabular}[\footnotesize]{|p{2.2cm}|p{2.2cm}|p{2.2cm}|p{1.1cm}|p{1.6cm}|p{0.8cm}|p{1cm}|p{0.8cm}|}
    \hline
        ~ & Cota & Cota tinctoria (L.) J.Gay & X & Abruzzo (ABAI/ABES); Campania (CAAI); Friuli (FRAI); Piemonte (PIAI/PIES); Sardegna (SAAI/SAES); Umbria (UMAI/UMES) & X & ~ & ~ \\ \hline
        ~ & Crepis & Crepis biennis L. & ~ & Friuli (FRES) & X & X & ~ \\ \hline
        ~ & ~ & Crepis sp. & ~ & Campania (CAAI/CAESD); Emilia-Romagna (ERAI/ERESP); Piemonte (PIAI/PIES); Puglia (PUES) & X & ~ & X \\ \hline
        ~ & ~ & Crepis taraxacifolia Thuill. & X & Friuli (FRAI) & X & ~ & ~ \\ \hline
        ~ & Cyanus & Cyanus segetum Hill. & X & Piemonte (PIES) & X & ~ & ~ \\ \hline
        ~ & Dittrichia & Dittrichia viscosa (L.) Greuter & X & Abruzzo (ABES); Sardegna (SAAI/SAES); Toscana (TOES) & X & ~ & ~ \\ \hline
        ~ & Erigeron  & Erigeron canadensis L. & X & Puglia (PUES) & X & ~ & ~ \\ \hline
        ~ & Galatella & Galatella linosyris (L.) Rchb. & X & Toscana (TOES) & X & ~ & ~ \\ \hline
        ~ & Glebionis & Glebionis coronaria (L.) Tzvelev & X & Puglia (PUAI); Sardegna (SAAI/SAES); Sicilia (SIAI) & X & ~ & ~ \\ \hline
        ~ & ~ & Glebionis segetum Fourr. & X & Campania (CAAI); Sardegna (SAES) & X & ~ & ~ \\ \hline
    \end{tabular}
    \end{table}
    
    \clearpage
        
        \begin{table}[!ht]
        \centering
    \begin{tabular}[\footnotesize]{|p{2.2cm}|p{2.2cm}|p{2.2cm}|p{1.1cm}|p{1.6cm}|p{0.8cm}|p{1cm}|p{0.8cm}|}
    \hline        
        ~ & Helminthotheca & Helminthotheca echioides (L.) Holub & X & Abruzzo (ABAI/ABES); Campania (CAESD); Sardegna (SAAI/SAES); Sicilia (SIAI/SIES); Umbria (UMAI) & X & ~ & ~ \\ \hline
        ~ & Hieracium & Hieracium glaucinum Jord. & ~ & Toscana (TOES) & X & X & ~ \\ \hline
        ~ & ~ & Hieracium sp. & ~ & Abruzzo (ABAI) & X & ~ & X \\ \hline
        ~ & Hypochaeris & Hypochaeris achyrophorus L. & X & Sardegna (SAAI/SAES) & X & ~ & ~ \\ \hline
        ~ & ~ & Hypochaeris radicata L. & X & Friuli (FRAI/FRES); Piemonte (PIAI); Sardegna (SAES) & X & ~ & ~ \\ \hline
        ~ & Inula & Inula sp. & ~ & Campania (CAESD) & X & ~ & X \\ \hline
        ~ & Jacobaea & Jacobaea delphiniifolia (Vahl) Pelser \& Veldkamp & X & Sardegna (SAAI) & X & ~ & ~ \\ \hline
        ~ & ~ & Jacobaea erucifolia  (L.) P.Gaertn., B.Mey. \& Schreb. & X & Abruzzo (ABAI/ABES) & X & ~ & ~ \\ \hline
        ~ & Lactuca & Lactuca sp. & ~ & Campania (CAAI/CAESD); Emilia-Romagna (ERAI/ERESP); Piemonte (PIAI/PIES) & X & ~ & X \\ \hline
        ~ & ~ & Lactuca viminea (L.) J.Presl \& C.Presl & X & Abruzzo (ABES) & X & ~ & ~ \\ \hline
        ~ & Scolymus & Scolymus grandiflorus Desf. & X & Sicilia (SIAI;SIES) & X & ~ & ~ \\ \hline
            \end{tabular}
    \end{table}
    
    \clearpage
        
        \begin{table}[!ht]
        \centering
    \begin{tabular}[\footnotesize]{|p{2.2cm}|p{2.2cm}|p{2.2cm}|p{1.1cm}|p{1.6cm}|p{0.8cm}|p{1cm}|p{0.8cm}|}
    \hline
        ~ & Sonchus & Sonchus sp. & ~ & Piemonte (PIAI/PIES) & X & ~ & X \\ \hline
        ~ & Tanacetum & Tanacetum sp. & ~ & Emilia-Romagna (ERESP) & X & ~ & X \\ \hline
        ~ & Taraxacum & Taraxacum campylodes G.E.Haglund & X & Campania (CAESD); Emilia-Romagna (ERAI/ERESP); Piemonte (PIAI/PIES); Umbria (UMAI;UMES) & X & ~ & ~ \\ \hline
        Convolvulaceae & Convolvulus  & Convolvulus cantabrica L. & X & Sardegna (SAES) & X & ~ & ~ \\ \hline
        ~ & ~ & Convolvulus sp. & X & Campania (CAAI/CAESD); Emilia-Romagna (ERAI/ERESP); Piemonte (PIAI;PIES) & X & ~ & X \\ \hline
        ~ & ~ & Convolvulus tricolor L. & X & Sicilia (SIAI/SIES) & X & ~ & ~ \\ \hline
        Cucurbitaceae & Bryonia & Bryonia cretica L. & X & Emilia-Romagna (ERAI) & X & ~ & ~ \\ \hline
        ~ & Ecballium & Ecballium elaterium (L.) A.Rich. & ~ & Puglia (PUAI); Sicilia (SIAI) & X & X & ~ \\ \hline
        Dioscoreaceae & Dioscorea & Dioscorea communis (L.) Caddick \& Wilkin & X & Sardegna (SAES) & X & ~ & ~ \\ \hline
        Euphorbiaceae & Euphorbia & Euphorbia insularis Boiss. & X & Sicilia (SIAI/SIES) & X & ~ & ~ \\ \hline
        ~ & Mercurialis  & Mercurialis sp. & ~ & Campania (CAAI) & X & ~ & X \\ \hline
        Gentianaceae & Centaurium & Centaurium maritimum Fritsch & X & Sardegna (SAAI/SAES) & X & ~ & ~ \\ \hline
            \end{tabular}
    \end{table}
    
    \clearpage
        
        \begin{table}[!ht]
        \centering
    \begin{tabular}[\footnotesize]{|p{2.2cm}|p{2.2cm}|p{2.2cm}|p{1.1cm}|p{1.6cm}|p{0.8cm}|p{1cm}|p{0.8cm}|}
    \hline
        ~ & ~ & Centaurium tenuiflorum (Hoffmanns. \& Link) Fritsch & ~ & Campania (CAAI); Sardegna (SAAI/SAES); Toscana (TOES); Umbria (UMAI) & X & X & ~ \\ \hline
        Geraniaceae & Erodium & Erodium moschatum (Burm.f.) L'Hér. & ~ & Sardegna (SAAI/SAES); Sicilia (SIAI/SIES) & X & X & ~ \\ \hline
        Iridaceae & Crocus & Crocus vernus (L.) Hill & X & Friuli (FRES) & X & ~ & ~ \\ \hline
        ~ & ~ & Crocus versicolor Ker Gawl. & ~ & Puglia (PUES) & X & X & ~ \\ \hline
        Lamiaceae & Ajuga & Ajuga iva (L.) Schreb. & ~ & Sardegna (SAAI) & X & X & ~ \\ \hline
        ~ & ~ & Ajuga sp. & X & Campania (CAESD) & X & ~ & X \\ \hline
        ~ & Clinopodium & Clinopodium nepeta (L.) Kuntze & X & Abruzzo (ABAI); Campania (CAAI/CAESD); Emilia-Romagna (ERAI/ERESP); Sardegna (SAAI); Umbria (UMES) & X & ~ & ~ \\ \hline
        ~ & Lamium & Lamium bifidum Cirillo & X & Abruzzo (ABAI) & X & ~ & ~ \\ \hline
        ~ & Mentha  & Mentha sp. & ~ & Campania (CAESD) & X & ~ & X \\ \hline
        Leguminosae & Bituminaria & Bituminaria bituminosa (L.) C.H.Stirt. & X & Abruzzo (ABES); Campania (CAESD) & X & ~ & ~ \\ \hline
        ~ & Dorycnium & Dorycnium hirsutum (L.) Ser. & X & Piemonte (PIES); Toscana (TOES) & X & ~ & ~ \\ \hline
        ~ & Hippocrepis & Hippocrepis emerus (L.) Lassen & X & Toscana (TOES) & X & ~ & ~ \\ \hline
        \end{tabular}
    \end{table}
    
    \clearpage
        
        \begin{table}[!ht]
        \centering
    \begin{tabular}[\footnotesize]{|p{2.2cm}|p{2.2cm}|p{2.2cm}|p{1.1cm}|p{1.6cm}|p{0.8cm}|p{1cm}|p{0.8cm}|}
    \hline
        ~ & Lotus & Lotus dorycnium L. & X & Campania (CAESD) & X & ~ & ~ \\ \hline
        ~ & ~ & Lotus herbaceus (Vill.) Jauzein & X & Abruzzo (ABES) & X & ~ & ~ \\ \hline
        ~ & ~ & Lotus sp.  & ~ & Piemonte (PIES) & X & ~ & X \\ \hline
        ~ & ~ & Lotus tetragonolobus L. & X & Sardegna (SAAI/SAES) & X & ~ & ~ \\ \hline
        ~ & Medicago & Medicago littoralis Rohde ex Loisel. & X & Sardegna (SAES) & X & ~ & ~ \\ \hline
        ~ & ~ & Medicago polymorpha L. & X & Abruzzo (ABAI/ABES); Sardegna (SAAI/SAES); Toscana (TOES); Umbria (UMAI) & X & ~ & ~ \\ \hline
            ~ & ~ & Medicago rugosa Desr. & X & Puglia (PUAI) & X & ~ & ~ \\ \hline
        ~ & ~ & Medicago scutellata (L.) Mill. & X & Abruzzo (ABAI); Puglia (PUES) & X & ~ & ~ \\ \hline
        ~ & Melilotus & Melilotus albus Medik. & X & Toscana (TOAI) & X & ~ & ~ \\ \hline
        ~ & ~ & Melilotus indicus (L.) All. & X & Puglia (PUAI); Sicilia (SIES); Toscana (TOAI) & X & ~ & ~ \\ \hline
        ~ & Securigera & Securigera varia (L.) Lassen & X & Emilia-Romagna (ERESP); Veneto (VEES) & X & ~ & ~ \\ \hline
        ~ & Sulla & Sulla coronaria (L.) Medik. & X & Abruzzo (ABAI/ABES) & X & ~ & ~ \\ \hline
        ~ & Trifolium & Trifolium sp. & ~ & Campania (CAESD) & X & ~ & ~ \\ \hline
        ~ & ~ & Trifolium subterraneum L. & ~ & Sardegna (SAAI/SAES) & X & X & ~ \\ \hline
        ~ & Trigonella & Trigonella altissima (Thuill.) Coulot \& Rabaute & X & Abruzzo (ABES) & X & ~ & ~ \\ \hline
        \end{tabular}
    \end{table}
    
    \clearpage
        
        \begin{table}[!ht]
        \centering
    \begin{tabular}[\footnotesize]{|p{2.2cm}|p{2.2cm}|p{2.2cm}|p{1.1cm}|p{1.6cm}|p{0.8cm}|p{1cm}|p{0.8cm}|}
    \hline
        ~ & Vicia & Vicia angustifolia L. & ~ & Friuli (FRAI/FRES) & X & X & ~ \\ \hline
        ~ & ~ & Vicia benghalensis L. & X & Sardegna (SAES) & X & ~ & ~ \\ \hline
        ~ & ~ & Vicia faba L. & X & Umbria (UMAI/UMES) & X & ~ & ~ \\ \hline
        ~ & ~ & Vicia sp. & ~ & Campania (CAESD); Sardegna (SAAI/SAES) & X & ~ & X \\ \hline
        Linaceae & Linum & Linum strictum L. & ~ & Sardegna (SAES) & X & X & ~ \\ \hline
        ~ & ~ & Linum trigynum L. & ~ & Abruzzo (ABES); Sardegna (SAES) & X & X & ~ \\ \hline
        ~ & ~ & Linum usitatissimum L. & X & Abruzzo (ABES) & X & ~ & ~ \\ \hline
        ~ & ~ & Linum usitatissimum L. subsp. angustifolium & X & Sardegna (SAES) & X & ~ & ~ \\ \hline
        Malvaceae & Malva & Malva multiflora (Cav.) Soldano, Banfi \& Galasso & X & Puglia (PUAI); Sardegna (SAAI/SAES) & X & ~ & ~ \\ \hline
        Oleaceae & Ligustrum & Ligustrum lucidum W.T. Aiton & X & Friuli (FRAI) & X & ~ & ~ \\ \hline
        ~ & Olea & Olea europaea L. & X & Sardegna (SAAI) & X & ~ & ~ \\ \hline
        Orobanchaceae & Bartsia & Bartsia trixago L. & ~ & Campania (CAESD) & X & X & ~ \\ \hline
        ~ & Phelipanche & Phelipanche sp. & X & Sardegna (SAES) & X & ~ & ~ \\ \hline
        Oxalidaceae & Oxalis & Oxalis corniculata L. & ~ & Campania (CAESD) & X & X & ~ \\ \hline
        ~ & ~ & Oxalis pes-caprae L. & X & Puglia (PUAI); Sardegna (SAES); Sicilia (SIAI/SIES) & X & ~ & ~ \\ \hline
        Papaveraceae & Fumaria & Fumaria bastardii Boreau & X & Puglia (PUAI/PUES) & X & ~ & ~ \\ \hline
        ~ & Papaver & Papaver apulum Ten. & X & Puglia (PUAI) & X & ~ & ~ \\ \hline
                    \end{tabular}
    \end{table}
    
    \clearpage
        
        \begin{table}[!ht]
        \centering
    \begin{tabular}[\footnotesize]{|p{2.2cm}|p{2.2cm}|p{2.2cm}|p{1.1cm}|p{1.6cm}|p{0.8cm}|p{1cm}|p{0.8cm}|}
    \hline
    ~ & ~ & Papaver dubium L. & X & Abruzzo (ABAI) & X & ~ & ~ \\ \hline
        ~ & ~ & Papaver hybridum L. & X & Puglia (PUES); Sardegna (SAAI) & X & ~ & ~ \\ \hline
        Plantaginaceae & Veronica & Veronica filiformis Sm. & ~ & Veneto (VEES) & X & X & ~ \\ \hline
        ~ & ~ & Veronica sp. & ~ & Emilia-Romagna (ERAI); Puglia (PUES) & X & ~ & X \\ \hline
        Polygonaceae & Persicaria & Persicaria lapathifolia (L.) Delarbre & X & Piemonte (PIES) & X & ~ & ~ \\ \hline
        ~ & ~ & Persicaria maculosa Gray & X & Umbria (UMAI) & X & ~ & ~ \\ \hline
        Primulaceae & Lysimachia & Lysimachia arvensis (L.) U.Manns \& Anderb. & X & Abruzzo (ABAI/ABES); Puglia (PUAI); Sardegna (SAAI/SAES) & X & ~ & ~ \\ \hline
        Ranunculaceae & Delphinium & Delphinium consolida L. & X & Abruzzo (ABAI) & X & ~ & ~ \\ \hline
        ~ & Ranunculus & Ranunculus ficaria L. & X & Abruzzo (ABAI/ABES); Campania (CAESD); Emilia-Romagna (ERESP); Piemonte (PIES); Sardegna (SAAI) & X & ~ & ~ \\ \hline
        Rosaceae & Malus & Malus sp. & ~ & Piemonte (PIES) & X & X & ~ \\ \hline
        ~ & Rosa & Rosa sp. & ~ & Piemonte (PIES) & X & ~ & X \\ \hline
        ~ & Rubus & Rubus sp. & ~ & Campania (CAAI/CAESD); Piemonte (PIES) & X & ~ & X \\ \hline
        Rubiaceae & Cruciata & Cruciata sp. & ~ & Campania (CAESD) & X & ~ & X \\ \hline
        Solanaceae & Solanum & Solanum americanum Mill. & X & Campania (CAAI/CAESD); Sicilia (SIAI) & X & ~ & ~ \\ \hline
    \end{tabular}
        \caption{flora entomofila rilevata nei siti BeeNet idonea ad essere erborizzata.}
    \label{tab:10}
\end{table}

\clearpage

Abbrev.: EG-NP = specie non presente in Erbario generale; AGR = regioni alle quali richiedere un campione; RC = richiesta di un campione; OOSG = presenza di un solo campione in Erbario Generale; SD = mancato riconoscimento dell’epiteto scientifico.

\clearpage

\end{document}





