\documentclass[main.tex]{subfiles}

\begin{document}

\subsection{Appendice 3}

\begin{table}[!ht]
    \centering
    \begin{tabular}[\footnotesize]{|p{2.4cm}|p{1.9cm}|p{5.8cm}|p{1.1cm}|p{1cm}|}
    \hline
        \textbf{Famiglia} & \textbf{Genere} & \textbf{Specie} & \textbf{ES-NP} & \textbf{OOSS} \\ \hline
        Apiaceae & Ammi & Ammi majus L. & X & ~ \\ \hline
        ~ & ~ & Ammi visnaga Gaertn. & X & ~ \\ \hline
        ~ & Daucus & Daucus carota L. & X & ~ \\ \hline
        ~ & Ferula & Ferula communis L. & X & ~ \\ \hline
        ~ & Foeniculum & Foeniculum vulgare Mill. & X & ~ \\ \hline
        ~ & Oenanthe & Oenanthe sp. & X & ~ \\ \hline
        ~ & Pimpinella & Pimpinella saxifraga L. & X & ~ \\ \hline
        ~ & Rouya & Rouya sp. & X & ~ \\ \hline
        ~ & Scandix & Scandix pecten-veneris L. & X & ~ \\ \hline
        ~ & ~ & Scandix sp.  & X & ~ \\ \hline
        ~ & Smyrnium & Smyrnium olusatrum L. & ~ & X \\ \hline
        ~ & Torilis & Torilis arvensis (Huds.) Link & X & ~ \\ \hline
        ~ & Visnaga & Visnaga daucoides Gaertn. & X & ~ \\ \hline
        Apocynaceae & Periploca & Periploca graeca L. & ~ & X \\ \hline
        ~ & Vinca & Vinca major L. & X & ~ \\ \hline
        Araceae & Arisarum & Arisarum vulgare O.Targ.Tozz. & X & ~ \\ \hline
        Asparagaceae & Bellevalia & Bellevalia romana Rchb. & X & ~ \\ \hline
        ~ & Loncomelos & Loncomelos brevistylum (Wolfner) Dostál & X & X \\ \hline
        ~ & Muscari & Muscari neglectum Guss. ex Ten. & X & ~ \\ \hline
        ~ & Ornithogalum & Ornithogalum comosum L. & X & X \\ \hline
        ~ & ~ & Ornithogalum divergens Boreau & X & ~ \\ \hline
        ~ & ~ & Ornithogalum umbellatum L. & X & ~ \\ \hline
        Asphodelaceae & Asphodeline & Asphodeline lutea Rchb. & X & ~ \\ \hline
        Boraginaceae & Buglossoides & Buglossoides arvensis (L.) I.M.Johnst. & X & ~ \\ \hline
        ~ & Cerinthe & Cerinthe major L. & X & X \\ \hline
        ~ & Cynoglossum & Cynoglossum creticum Mill. & X & ~ \\ \hline
        ~ & Myosotis & Myosotis ramosissima Rochel & X & ~ \\ \hline
        Brassicaceae & Arabidopsis & Arabidopsis sp. & X & ~ \\ \hline
        ~ & Brassica & Brassica nigra W.D.J.Koch & X & ~ \\ \hline
        ~ & ~ & Brassica rapa L. & X & ~ \\ \hline
        ~ & ~ & Brassica sp. & X & ~ \\ \hline
        ~ & Calepina & Calepina irregularis Thell. & X & ~ \\ \hline
        ~ & Capsella & Capsella bursa-pastoris Medik. & X & ~ \\ \hline
        ~ & Cardamine & Cardamine flexuosa With. & X & ~ \\ \hline
        ~ & ~ & Cardamine sp. & X & ~ \\ \hline
        ~ & Erucastrum  & Erucastrum incanum Koch. & X & ~ \\ \hline
        ~ & Lepidium & Lepidium draba L. & ~ & X \\ \hline
        ~ & Lobularia & Lobularia maritima (L.) Desv. & X & ~ \\ \hline
        ~ & Microthlaspi & Microthlaspi perfoliatum (L.) F.K. Mey. & X & ~ \\ \hline
        ~ & Rorippa & Rorippa sylvestris (L.) Besser & X & ~ \\ \hline
        ~ & Sinapis & Sinapis sp. & X & ~ \\ \hline
        Campanulaceae & Legousia & Legousia speculum-veneris (L.) Chaix & X & ~ \\ \hline
        Caprifoliaceae & Cephalaria & Cephalaria transsylvanica L. & X & ~ \\ \hline
        ~ & Knautia & Knautia arvensis J.M.Coult. & X & ~ \\ \hline
        ~ & Scabiosa & Scabiosa atropurpurea L. & X & ~ \\ \hline
        ~ & ~ & Scabiosa columbaria L. & X & ~ \\ \hline
        ~ & Valerianella & Valerianella locusta L. & X & ~ \\ \hline
        Caryophyllaceae & Cerastium & Cerastium glomeratum Thuill. & X & ~ \\ \hline
        ~ & Paronychia & Paronychia sp. & X & ~ \\ \hline
            \end{tabular}
    \end{table}
    
    \clearpage
        
        \begin{table}[!ht]
        \centering
    \begin{tabular}[\footnotesize]{|p{2.4cm}|p{1.9cm}|p{5.8cm}|p{1.1cm}|p{1cm}|}
    \hline
        ~ & Petrorhagia & Petrorhagia dubia Raf. & X & ~ \\ \hline
        ~ & ~ & Petrorhagia prolifera (L.) P.W.Ball \& Heywood & X & ~ \\ \hline
        ~ & ~ & Petrorhagia saxifraga Link & X & ~ \\ \hline
        ~ & Silene & Silene dioica (L.) Clairv. & X & ~ \\ \hline
        ~ & ~ & Silene latifolia Poir. & X & ~ \\ \hline
        ~ & ~ & Silene vulgaris (Moench) Garcke & X & ~ \\ \hline
        Celastraceae & Euonymus & Euonymus europaeus L. & X & ~ \\ \hline
        Chenopodiaceae & Atriplex & Atriplex prostrata Boucher ex DC. & X & ~ \\ \hline
        Compositae & Achillea & Achillea roseoalba Ehrend. & X & ~ \\ \hline
        ~ & Ambrosia & Ambrosia artemisiifolia L. & X & ~ \\ \hline
        ~ & ~ & Ambrosia psilostachya DC. & X & ~ \\ \hline
        ~ & Anacyclus & Anacyclus clavatus (Desf.) Pers. & X & ~ \\ \hline
        ~ & ~ & Anacyclus radiatus Loisel. & X & ~ \\ \hline
        ~ & Anthemis & Anthemis sp. & X & ~ \\ \hline
        ~ & Aster & Aster subulatus Michx. & X & X \\ \hline
        ~ & Bellis & Bellis sp. & X & X \\ \hline
        ~ & Bidens & Bidens frondosa L. & X & ~ \\ \hline
        ~ & Calendula & Calendula sp. & X & ~ \\ \hline
        ~ & Carthamus & Carthamus caeruleus L & X & ~ \\ \hline
        ~ & ~ & Carthamus lanatus L. & X & ~ \\ \hline
        ~ & Centaurea & Centaurea calcitrapa L. & X & ~ \\ \hline
        ~ & ~ & Centaurea cyanus L. & X & ~ \\ \hline
        ~ & ~ & Centaurea diluta Dryand. & X & ~ \\ \hline
        ~ & ~ & Centaurea napifolia L. & X & ~ \\ \hline
        ~ & ~ & Centaurea nigrescens Willd. & X & ~ \\ \hline
        ~ & Chondrilla & Chondrilla juncea L. & X & ~ \\ \hline
        ~ & Cichorium & Cichorium intybus L. & X & ~ \\ \hline
        ~ & Cirsium & Cirsium arvense (L.) Scop. & X & ~ \\ \hline
        ~ & ~ & Cirsium oleraceum Scop. & X & ~ \\ \hline
        ~ & ~ & Cirsium tenoreanum Petr. & X & X \\ \hline
        ~ & ~ & Cirsium vulgare (Savi) Ten. & X & ~ \\ \hline
        ~ & Cota & Cota tinctoria (L.) J.Gay & X & ~ \\ \hline
        ~ & Crepis & Crepis biennis L. & X & X \\ \hline
        ~ & ~ & Crepis capillaris (L.) Wallr. & X & ~ \\ \hline
        ~ & ~ & Crepis rubra L. & X & ~ \\ \hline
        ~ & ~ & Crepis sancta (L.) Babc. & X & ~ \\ \hline
        ~ & ~ & Crepis setosa Haller f. & X & ~ \\ \hline
        ~ & ~ & Crepis sp. & X & ~ \\ \hline
        ~ & ~ & Crepis taraxacifolia Thuill. & X & ~ \\ \hline
        ~ & ~ & Crepis vesicaria L. & X & ~ \\ \hline
        ~ & Cyanus & Cyanus segetum Hill. & X & ~ \\ \hline
        ~ & Cynara & Cynara cardunculus L. & X & ~ \\ \hline
        ~ & Dittrichia & Dittrichia viscosa (L.) Greuter & X & ~ \\ \hline
        ~ & Erigeron & Erigeron annuus (L.) Pers. & X & ~ \\ \hline
        ~ & Galactites & Galactites tomentosus Moench & X & ~ \\ \hline
        ~ & Galatella & Galatella linosyris (L.) Rchb. & X & ~ \\ \hline
        ~ & Galinsoga & Galinsoga parviflora Cav. & X & ~ \\ \hline
        ~ & Glebionis & Glebionis coronaria (L.) Tzvelev & X & ~ \\ \hline
        ~ & ~ & Glebionis segetum Fourr. & X & ~ \\ \hline
        ~ & Helianthus & Helianthus annuus L. & X & ~ \\ \hline
        ~ & Helminthotheca & Helminthotheca echioides & X & ~ \\ \hline
        ~ & Hieracium & Hieracium glaucinum Jord. & X & X \\ \hline
        ~ & ~ & Hieracium sp. & X & ~ \\ \hline
        ~ & Hypochaeris & Hypochaeris achyrophorus L. & X & ~ \\ \hline
        ~ & ~ & Hypochaeris radicata L. & X & ~ \\ \hline
       \end{tabular}
    \end{table}
    
    \clearpage
        
        \begin{table}[!ht]
        \centering
    \begin{tabular}[\footnotesize]{|p{2.4cm}|p{1.9cm}|p{5.8cm}|p{1.1cm}|p{1cm}|}
    \hline
        ~ & Inula & Inula conyza DC. & X & ~ \\ \hline
        ~ & ~ & Inula sp. & X & ~ \\ \hline
        ~ & Jacobaea & Jacobaea delphiniifolia (Vahl) Pelser \& Veldkamp & X & ~ \\ \hline
        ~ & ~ & Jacobaea erucifolia (L.) P.Gaertn., B.Mey. \& Schreb. & X & ~ \\ \hline
        ~ & Lactuca & Lactuca saligna L. & X & ~ \\ \hline
        ~ & ~ & Lactuca sp. & X & ~ \\ \hline
        ~ & ~ & Lactuca viminea (L.) J.Presl \& C.Presl & X & ~ \\ \hline
        ~ & Leontodon & Leontodon hispidus L. & X & ~ \\ \hline
        ~ & Leucanthemum & Leucanthemum sp. & X & ~ \\ \hline
        ~ & ~ & Leucanthemum vulgare Lam. & X & ~ \\ \hline
        ~ & Pallenis & Pallenis spinosa (L.) Cass. & X & ~ \\ \hline
        ~ & Pulicaria & Pulicaria dysenterica Gaertn. & X & ~ \\ \hline
        ~ & ~ & Pulicaria vulgaris Gaertn. & X & ~ \\ \hline
        ~ & Reichardia & Reichardia picroides (L.) Roth & X & ~ \\ \hline
        ~ & Rhagadiolus & Rhagadiolus stellatus (L.) Gaertn. & X & ~ \\ \hline
        ~ & Scolymus & Scolymus grandiflorus Desf. & X & ~ \\ \hline
        ~ & ~ & Scolymus hispanicus L. & X & ~ \\ \hline
        ~ & Senecio & Senecio squalidus L. & X & ~ \\ \hline
        ~ & ~ & Senecio vulgaris L. & X & ~ \\ \hline
        ~ & Silybum & Silybum marianum (L.) Gaertn. & X & ~ \\ \hline
        ~ & ~ & Sonchus sp. & X & ~ \\ \hline
        ~ & Tanacetum & Tanacetum sp. & X & ~ \\ \hline
        ~ & Taraxacum & Taraxacum campylodes G.E.Haglund & X & ~ \\ \hline
        ~ & ~ & Taraxacum officinale F.H.Wigg. & X & ~ \\ \hline
        ~ & ~ & Taraxacum sect. Taraxacum  & X & ~ \\ \hline
        Convolvulaceae & Calystegia & Calystegia sepium (L.) R.Br. & X & ~ \\ \hline
        ~ & Convolvulus & Convolvulus sp. & X & ~ \\ \hline
        Cucurbitaceae & Bryonia & Bryonia alba L. & X & ~ \\ \hline
        ~ & ~ & Bryonia cretica L. & X & ~ \\ \hline
        ~ & Ecballium & Ecballium elaterium (L.) A.Rich. & X & X \\ \hline
        Dioscoreaceae & Dioscorea & Dioscorea communis (L.) Caddick \& Wilkin & X & ~ \\ \hline
        Dipsacaceae & Sixalix & Sixalix atropurpurea L. & X & ~ \\ \hline
        Euphorbiaceae & Chrozophora & Chrozophora tinctoria (L.) A.Juss. & X & ~ \\ \hline
        ~ & Euphorbia & Euphorbia insularis Boiss. & X & ~ \\ \hline
        ~ & ~ & Euphorbia pithyusa L. & X & ~ \\ \hline
        ~ & Mercurialis & Mercurialis sp. & X & X \\ \hline
        Fagaceae & Quercus & Quercus ilex & X & ~ \\ \hline
        Gentianaceae & Blackstonia & Blackstonia perfoliata (L.) Huds. & X & ~ \\ \hline
        ~ & Centaurium & Centaurium erythraea Rafn & X & ~ \\ \hline
        ~ & ~ & Centaurium maritimum Fritsch & X & ~ \\ \hline
        ~ & ~ & Centaurium pulchellum (Sw.) Druce & X & ~ \\ \hline
        ~ & ~ & Centaurium tenuiflorum (Hoffmanns. \& Link) Fritsch & X & X \\ \hline
        Geraniaceae & Erodium & Erodium moschatum (Burm.f.) L'Hér. & ~ & X \\ \hline
        Hypericaceae & Hypericum & Hypericum humifusum L. & X & ~ \\ \hline
        ~ & ~ & Hypericum perfoliatum L. & X & ~ \\ \hline
        ~ & ~ & Hypericum perforatum L. & X & ~ \\ \hline
        ~ & ~ & Hypericum triquetrifolium Turra & X & ~ \\ \hline
        Iridaceae & Crocus & Crocus versicolor Ker Gawl. & ~ & X \\ \hline
        ~ & Gladiolus & Gladiolus italicus Mill. & X & ~ \\ \hline
        Lamiaceae & Ajuga & Ajuga iva (L.) Schreb. & ~ & X \\ \hline
        ~ & ~ & Ajuga sp. & X & ~ \\ \hline
        ~ & Clinopodium & Clinopodium nepeta (L.) Kuntze & X & ~ \\ \hline
       \end{tabular}
    \end{table}
    
    \clearpage
        
        \begin{table}[!ht]
        \centering
    \begin{tabular}[\footnotesize]{|p{2.4cm}|p{1.9cm}|p{5.8cm}|p{1.1cm}|p{1cm}|}
    \hline
        ~ & Mentha & Mentha longifolia L. & X & ~ \\ \hline
        ~ & ~ & Mentha sp. & X & ~ \\ \hline
        ~ & ~ & Mentha spicata L. & X & ~ \\ \hline
        ~ & ~ & Mentha suaveolens Ehrh. & X & ~ \\ \hline
        ~ & Salvia & Salvia clandestina L. & X & ~ \\ \hline
        Leguminosae & Astragalus & Astragalus hamosus L. & X & ~ \\ \hline
        ~ & ~ & Astragalus monspessulanus L. & X & ~ \\ \hline
        ~ & Bituminaria & Bituminaria bituminosa (L.) C.H.Stirt. & X & ~ \\ \hline
        ~ & Coronilla & Coronilla scorpioides (L.) W.D.J.Koch & X & ~ \\ \hline
        ~ & Dorycnium & Dorycnium hirsutum (L.) Ser. & X & ~ \\ \hline
        ~ & Hippocrepis & Hippocrepis emerus (L.) Lassen & X & ~ \\ \hline
        ~ & Lathyrus & Lathyrus annuus L. & X & ~ \\ \hline
        ~ & ~ & Lathyrus pratensis L. & X & ~ \\ \hline
        ~ & ~ & Lathyrus sp. (bianco) & X & ~ \\ \hline
        ~ & Lotus & Lotus dorycnium L. & X & ~ \\ \hline
        ~ & ~ & Lotus herbaceus (Vill.) Jauzein & X & ~ \\ \hline
        ~ & ~ & Lotus pedunculatus Cav. & X & ~ \\ \hline
        ~ & ~ & Lotus sp.  & X & ~ \\ \hline
        ~ & Medicago & Medicago arabica (L.) Huds. & X & ~ \\ \hline
        ~ & ~ & Medicago littoralis Rohde ex Loisel. & X & ~ \\ \hline
        ~ & ~ & Medicago minima (L.) Bartal. & X & ~ \\ \hline
        ~ & ~ & Medicago polymorpha L. & X & ~ \\ \hline
        ~ & ~ & Medicago rugosa Desr. & X & ~ \\ \hline
        ~ & ~ & Medicago truncatula Gaertn. & X & ~ \\ \hline
        ~ & Melilotus & Melilotus albus Medik. & X & ~ \\ \hline
        ~ & ~ & Melilotus indicus (L.) All. & X & ~ \\ \hline
        ~ & Onobrychis & Onobrychis viciifolia Scop. & X & ~ \\ \hline
        ~ & Robinia & Robinia pseudoacacia L. & X & ~ \\ \hline
        ~ & Securigera & Securigera varia (L.) Lassen & X & ~ \\ \hline
        ~ & Spartium & Spartium junceum L. & X & ~ \\ \hline
        ~ & Sulla & Sulla coronaria (L.) Medik. & X & ~ \\ \hline
        ~ & Trifolium & Trifolium campestre Schreb. & X & ~ \\ \hline
        ~ & ~ & Trifolium sp. & X & ~ \\ \hline
        ~ & ~ & Trifolium subterraneum L. & ~ & X \\ \hline
        ~ & Trigonella & Trigonella altissima (Thuill.) Coulot \& Rabaute & X & ~ \\ \hline
        ~ & Vicia & Vicia angustifolia L. & ~ & X \\ \hline
        ~ & ~ & Vicia benghalensis L & X & ~ \\ \hline
        ~ & ~ & Vicia bithynica L. & X & ~ \\ \hline
        ~ & ~ & Vicia faba L. & X & ~ \\ \hline
        ~ & ~ & Vicia sp. & X & ~ \\ \hline
        Liliaceae & Gagea & Gagea lutea Ker Gawl. & X & ~ \\ \hline
        Linaceae & Linum & Linum strictum L. & ~ & X \\ \hline
        ~ & ~ & Linum trigynum L. & X & X \\ \hline
        ~ & ~ & Linum usitatissimum L. subsp. Angustifolium & X & ~ \\ \hline
        Lythraceae & Lythrum & Lythrum junceum Banks ex Sol. & X & ~ \\ \hline
        Malvaceae & ~ & Malva multiflora (Cav.) Soldano, Banfi \& Galasso & X & ~ \\ \hline
        ~ & ~ & Malva neglecta Wallr. & X & ~ \\ \hline
        Oleaceae & Ligustrum & Ligustrum lucidum W.T. Aiton & X & ~ \\ \hline
        Orchidaceae & Anacamptis & Anacamptis pyramidalis (L.) Rich. & X & ~ \\ \hline
        ~ & Orchis & Orchis purpurea Huds. & X & ~ \\ \hline
        Orobanchaceae & Bartsia & Bartsia trixago L. & ~ & X \\ \hline
        ~ & Bellardia & Bellardia trixago (L.) All. & X & ~ \\ \hline
        ~ & ~ & Bellardia viscosa (L.) Fisch. \& C.A.Mey. & X & ~ \\ \hline
        \end{tabular}
    \end{table}
    
    \clearpage
        
        \begin{table}[!ht]
        \centering
    \begin{tabular}[\footnotesize]{|p{2.4cm}|p{1.9cm}|p{5.8cm}|p{1.1cm}|p{1cm}|}
    \hline
        ~ & Orobanche & Orobanche sp. & X & ~ \\ \hline
        ~ & Parentucellia & Parentucellia viscosa (L.) Caruel & X & ~ \\ \hline
        ~ & Phelipanche & Phelipanche sp. & X & ~ \\ \hline
        Oxalidaceae & Oxalis & Oxalis corniculata L & ~ & X \\ \hline
        ~ & ~ & Oxalis dillenii Jacq. & X & ~ \\ \hline
        ~ & ~ & Oxalis pes-caprae L. & X & ~ \\ \hline
        Papaveraceae & Fumaria & Fumaria bastardii Boreau & X & ~ \\ \hline
        Phytolaccaceae & Phytolacca & Phytolacca americana L. & X & ~ \\ \hline
        Plantaginaceae & Kickxia & Kickxia commutata (Bernh. ex Rchb.) Fritsch & X & ~ \\ \hline
        ~ & ~ & Kickxia spuria (L.) Dumort. & X & ~ \\ \hline
        ~ & Linaria & Linaria vulgaris Mill. & X & ~ \\ \hline
        ~ & Plantago & Plantago lagopus L. & X & ~ \\ \hline
        ~ & ~ & Plantago lanceolata L. & X & ~ \\ \hline
        ~ & ~ & Plantago major L. & X & ~ \\ \hline
        ~ & Veronica & Veronica filiformis Sm. & X & X \\ \hline
        ~ & ~ & Veronica persica Poir. & X & ~ \\ \hline
        ~ & ~ & Veronica sp. & X & ~ \\ \hline
        Polygonaceae & Persicaria & Persicaria lapathifolia (L.) Delarbre & X & ~ \\ \hline
        ~ & ~ & Persicaria maculosa Gray & X & ~ \\ \hline
        Primulaceae & Lysimachia & Lysimachia arvensis (L.) U.Manns \& Anderb. & X & ~ \\ \hline
        ~ & Primula & Primula vulgaris Huds. & X & ~ \\ \hline
        Ranunculaceae & Anemone & Anemone hortensis L. & X & ~ \\ \hline
        ~ & Clematis & Clematis flammula L. & X & ~ \\ \hline
        ~ & ~ & Clematis vitalba L. & X & ~ \\ \hline
        ~ & Ranunculus & Ranunculus ficaria L. & X & ~ \\ \hline
        Resedaceae & Reseda & Reseda alba L. & X & ~ \\ \hline
        Rhamnaceae & Rhamnus & Rhamnus alaternus L. & X & ~ \\ \hline
        ~ & Paliurus & Paliurus spina-christi Mill. & X & ~ \\ \hline
        Rosaceae & Crataegus & Crataegus monogyna Jacq. & X & ~ \\ \hline
        ~ & Filipendula & Filipendula ulmaria (L.) Maxim. & X & ~ \\ \hline
        ~ & Malus & Malus sp. & X & X \\ \hline
        ~ & Prunus & Prunus avium L. & X & ~ \\ \hline
        ~ & Pyrus & Pyrus communis L. & X & ~ \\ \hline
        ~ & Rosa & Rosa canina L. & X & ~ \\ \hline
        ~ & ~ & Rosa sp. & X & ~ \\ \hline
        ~ & Rubus & Rubus sp. & X & ~ \\ \hline
        ~ & ~ & Rubus ulmifolius Schott & X & ~ \\ \hline
        Rubiaceae & Cruciata & Cruciata sp. & X & ~ \\ \hline
        ~ & Galium & Galium album Mill. & X & ~ \\ \hline
        ~ & ~ & Galium aparine L. & X & ~ \\ \hline
        ~ & ~ & Galium mollugo L. & X & ~ \\ \hline
        ~ & ~ & Galium verum L. & X & ~ \\ \hline
        ~ & Rubia & Rubia peregrina L. & X & ~ \\ \hline
        ~ & Sherardia & Sherardia arvensis L. & X & ~ \\ \hline
        Solanaceae & Solanum & Solanum americanum Mill. & X & ~ \\ \hline
        ~ & ~ & Solanum nigrum L. & X & ~ \\ \hline
        Valerianaceae & Centranthus & Centranthus calcitrapa L. & X & ~ \\ \hline
        Violaceae & Viola & Viola arvensis Murray & X \\ \hline
    \end{tabular}
        \caption{Specie vegetali rilevate nei siti BeeNet e presenti nell’erbario storico di A. Bertoloni.}
    \label{tab:12}
\end{table}

Abbrev.: ES-NP = specie non presente in Erbario Storico; OOSS = presenza di un solo campione in Erbario Storico.

\clearpage

\end{document}





