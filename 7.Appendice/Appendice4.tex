\documentclass[main.tex]{subfiles}

\begin{document}

\subsection{Appendice 3}

\begin{table}[!ht]
    \centering
    \begin{tabular}[\footnotesize]{|p{3cm}|p{9cm}|}
    \hline
        \textbf{Famiglia} & \textbf{Generi} \\ \hline
        Adoxaceae & Sambucus \\ \hline
        Amaranthaceae & Amaranthus, Chenopodium \\ \hline
        Amaryllidaceae & Allium, Narcissus \\ \hline
        Apiaceae & Ammi, Daucus, Eryngium, Ferula, Foeniculum, Heracleum, Oenanthe, Pastinaca, Pimpinella, Rouya, Scandix, Smyrnium, Tordylium, Torilis, Visnaga \\ \hline
        Apocynaceae & Periploca, Vinca \\ \hline
        Araceae  & Arisarum, Arum \\ \hline
        Araliaceae & Hedera \\ \hline
        Asparagaceae & Asparagus, Bellevalia, Loncomelos, Muscari, Ornithogalum \\ \hline
        Asphodelaceae & Asphodeline \\ \hline
        Boraginaceae & Anchusa, Borago, Buglossoides, Cerinthe, Cynoglossum, Echium, Myosotis, Symphytum \\ \hline
        Brassicaceae & Arabidopsis, Barbarea, Brassica, Calepina, Capsella, Cardamine, Diplotaxis, Draba, Erucastrum, Isatis, Lepidium, Lobularia, Microthlaspi, Raphanus, Rapistrum, Rorippa, Sinapis \\ \hline
        Campanulaceae & Campanula, Legousia \\ \hline
        Caprifoliaceae & Cephalaria, Dipsacus, Knautia, Scabiosa, Valerianella, Viburnum \\ \hline
        Caryophyllaceae & Cerastium, Dianthus, Lychnis, Paronychia, Petrorhagia, Saponaria, Silene, Stellaria \\ \hline
        Celastraceae & Euonymus \\ \hline
        Chenopodiaceae & Atriplex \\ \hline
        Cistaceae & Cistus \\ \hline
        Colchicaceae & Colchicum \\ \hline
        Compositae & Achillea, Ambrosia, Anacyclus, Andryala, Anthemis, Artemisia, Aster, Bellis, Bidens, Calendula, Carduus, Carlina, Carthamus, Centaurea, Chondrilla, Cichorium, Cirsium, Cota, Crepis, Cyanus, Cynara, Dittrichia, Erigeron, Eupatorium, Galactites, Galatella, Galinsoga, Glebionis, Hedypnois, Helianthus, Helminthotheca, Hieracium, Hyoseris, Hypochaeris, Inula, Jacobaea, Lactuca, Leontodon, Leucanthemum, Matricaria, Pallenis, Picris, Pulicaria, Reichardia, Rhagadiolus, Scolymus, Senecio, Silybum, Sonchus, Tanacetum, Taraxacum, Tragopogon, Urospermum \\ \hline
        Convolvulaceae & Calystegia, Convolvulus \\ \hline
        Cornaceae & Cornus \\ \hline
        Cucurbitaceae & Bryonia, Ecballium \\ \hline
        Dioscoreaceae & Dioscorea \\ \hline
        Dipsacaceae & Sixalix \\ \hline
        Ericaceae & Erica \\ \hline
        Euphorbiaceae & Chrozophora, Euphorbia, Mercurialis \\ \hline
        Fagaceae & Quercus \\ \hline
        Gentianaceae & Blackstonia, Centaurium \\ \hline
        Geraniaceae & Erodium, Geranium \\ \hline
        Hypericaceae & Hypericum \\ \hline
        Iridaceae & Crocus, Gladiolus \\ \hline
        Lamiaceae & Ajuga, Ballota, Clinopodium, Glechoma, Lamium, Mentha, Prunella, Salvia, Stachys \\ \hline
        \end{tabular}
    \end{table}
    
    \clearpage
        
        \begin{table}[!ht]
        \centering
    \begin{tabular}[\footnotesize]{|p{3cm}|p{9cm}|}
    \hline 
        Leguminosae & Astragalus, Bituminaria, Cercis, Coronilla, Dorycnium, Glycyrrhiza, Hedysarum, Hippocrepis, Lathyrus, Lotus, Medicago, Melilotus, Onobrychis, Robinia, Scorpiurus, Securigera, Spartium, Sulla, Trifolium, Trigonella, Vicia \\ \hline
        Liliaceae & Gagea \\ \hline
        Linaceae & Linum \\ \hline
        Lythraceae & Lythrum \\ \hline
        Malvaceae & Althaea, Malope, Malva \\ \hline
        Myrtaceae & Myrtus \\ \hline
        Oleaceae & Ligustrum, Olea \\ \hline
        Onagraceae & Epilobium, Oenothera \\ \hline
        Orchidaceae & Anacamptis, Orchis \\ \hline
        Orobanchaceae & Bartsia, Bellardia, Orobanche, Parentucellia, Phelipanche \\ \hline
        Oxalidaceae & Oxalis \\ \hline
        Papaveraceae & Fumaria, Papaver \\ \hline
        Phytolaccaceae & Phytolacca \\ \hline
        Pinaceae & Pinus \\ \hline
        Plantaginaceae & Kickxia, Linaria, Plantago, Plumbago, Veronica \\ \hline
        Poaceae & Dactylis \\ \hline
        Polygonaceae & Persicaria, Rumex \\ \hline
        Portulacaceae & Portulaca \\ \hline
        Primulaceae & Anagallis, Lysimachia, Primula \\ \hline
        Ranunculaceae & Anemone, Clematis, Delphinium, Ranunculus, Nigella, Ranunculus \\ \hline
        Resedaceae & Reseda \\ \hline
        Rhamnaceae & Rhamnus, Paliurus \\ \hline
        Rosaceae & Agrimonia, Crataegus, Filipendula, Fragaria, Geum, Malus, Potentilla, Poterium, Prunus, Pyrus, Rosa, Rubus, Sanguisorba \\ \hline
        Rubiaceae & Cruciata, Galium, Rubia, Sherardia \\ \hline
        Scrophulariaceae & Verbascum \\ \hline
        Solanaceae & Solanum \\ \hline
        Tamaricaceae & Tamarix \\ \hline
        Valerianaceae & Centranthus, Valeriana \\ \hline
        Verbenaceae & Verbena \\ \hline
        Violaceae & Viola \\ \hline
        Xanthorrhoeaceae & Asphodelus \\ \hline
        Zygophyllaceae & Tribulus \\ \hline
    \end{tabular}
        \caption{famiglie e generi tassonomici della flora entomofila rilevata nei siti BeeNet.}
    \label{tab:13}
\end{table}


\end{document}





