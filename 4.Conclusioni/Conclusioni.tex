\documentclass[main.tex]{subfiles}

\begin{document}

\section{DISCUSSIONE E CONCLUSIONI}
\subsection{Considerazioni conclusive}

\textnormal{
Questo studio ha evidenziato l'importanza di considerare le collezioni di exsiccata in relazione al network ecologico composto da piante ed impollinatori ubicato nei rispettivi siti di studio BeeNet. \\
In particolare, il presente studio ha definito la mancanza di 76 specie vegetali nella collezione dell’Erbario Generale ed ha consentito di verificare la presenza di specie oggi considerate rare ed endemiche nell’Erbario Storico di Antonio Bertoloni. In questo modo, oltre a definire le specie idonee per nuove erborizzazioni secondo i criteri prestabiliti, è stato possibile constatare che tutte le specie rare ed endemiche presenti nella Flora italica sono state rilevate durante le identificazioni della flora entomofila nei siti BeeNet. Determinando, dunque, che dalla sua pubblicazione non sono avvenuti cambiamenti floristici \citep{aless} dovuti alla scomparsa di specie oggi considerate rare ed endemiche. Sebbene la maggior parte delle specie considerate sia presente in erbario ritengo importante implementare la collezione, per avere altri campioni di confronto. \\
Considerato il grado di similarità floristica (al rango di genere) tra i siti intensivi (ABAI, CAAI, ERAI, FRAI, PIAI, PUAI, SAAI, SIAI, TOAI, UMAI, VEAI) e seminaturali (ABES, CAESD, ERESP, FRES, PIES, PUES, SAES, SIES, TOES, UMES, VEES) l’indice di Sørensen denota un alto numero di generi differenti e ciò constata che i siti intensivi hanno una minor ricchezza tassonomica rispetto a quelli seminaturali. Mentre la macroregione del CENTRO dimostra di avere un maggiore numero di generi in comune rispetto agli agroecosistemi delle altre macroregioni rivelando un minore diversità. I presenti indici sono utili nel rivelare l’impatto dell’agricoltura sui presenti agroecosistemi, definendo così il loro stato di salute. \\
L’analisi dei network nei siti ERAI/ERESP della regione Emilia-Romagna, in particolare il calcolo della modularità e dell’indice di Fisher alpha, indica che la struttura delle comunità piante-impollinatori è maggiormente articolata nei siti seminaturali. Dunque, posso ipotizzare che le comunità di impollinatori necessitano di un ambiente più eterogeneo in grado di sostenerle sia dal punto di vista delle risorse trofiche sia per la più ampia disponibilità di siti di nidificazione non disturbati. \\
Le reti mutualistiche sono fortemente soggette alle modifiche ambientali, prevalentemente causate dalle azioni antropiche, e la stabilità delle reti di impollinazione può subire impatti negativi quando la maggior parte delle specie presentano connessioni reciproche che si interrompono. Dunque, è di assoluta importanza la tutela degli impollinatori e della relativa flora entomofila al fine di evitare il collasso delle interazioni tra piante-impollinatori e preservare la diversità funzionale della comunità.
}

% Pagina vuota
\clearpage
\null
\thispagestyle{empty}
\clearpage

\end{document}





