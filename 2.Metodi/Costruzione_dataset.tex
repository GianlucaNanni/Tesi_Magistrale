\documentclass[main.tex]{subfiles}

\begin{document}

\subsection{Costruzione del dataset}

I dati che vanno a costituire il dataset derivano dall’identificazione e dal conteggio delle specie vegetali presenti nei transetti dei siti di studio. Per database si intende un archivio di dati strutturato in maniera tale da consentire la gestione e l'aggiornamento delle informazioni, l’organizzazione in formato tabellare permette di effettuare aggiornamenti e ricerche incrociate \citep{satta}.
Il dataset in oggetto contiene la flora entomofila rilevata nei siti di studio BeeNet. Tale flora è stata identificata a livello di specie da personale opportunamente formato ed ha la finalità di descrivere la componente vegetale di interesse per gli insetti pronubi presente negli agroecosistemi intensivi e seminaturali.
Il dataset, sotto forma di tabella (Tab. \ref{tab:2}), comprende l’identificazione tassonomica della flora comprensiva di famiglia/genere/specie, il sito regionale dove è stata rilevata e la tipologia di agroecosistema, il numero di specie parziale per ogni sito e totale per regione.

\begin{table}[h!]
    \centering
\begin{tabular}[\small]{|1|1|1|1|1|1|}
\hline
Famiglia & Genere & Specie & Sito/Agroecosistema & Totale sp. parziale & Tot. sp. complessivo\\
\hline
\end{tabular}
    \caption{caratteristiche del dataset contenente la flora entomofila BeeNet.}
    \label{tab:2}
\end{table}

\end{document}