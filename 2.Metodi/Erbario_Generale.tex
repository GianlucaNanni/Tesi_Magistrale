\documentclass[main.tex]{subfiles}

\begin{document}

\subsection{Analisi tra flora rilevata nei siti di studio e l’Erbario Generale di BRAHMS BOLO}

BRAHMS è un software avente la funzione di database gestionale per la cura delle raccolte di storia naturale, banche del seme, orti botanici, rilievi sul campo, studi biogeografici e ricerche tassonomiche \href{https://herbaria.plants.ox.ac.uk/bol/}{(BRAHMS DATABASE, 2022)}. L’analisi dei dati e l’attività di ricerca che tale software consente di effettuare rende possibile analizzare dati e immagini creando rapporti e mappe.
L’attività principale per cui viene impiegato BRAHMS per gli erbari è la gestione delle collezioni, poiché rende possibile la catalogazione di ogni tipologia di campione indipendentemente che si tratti di fogli di erbario, campioni di legno, DNA voucher, collezioni sottovetro o altro.
Nella presente analisi si fa riferimento alla collezione di piante essiccate presente sul database BRAHMS BOLO, costituita da oltre 52.000 campioni e realizzata tramite la digitalizzazione di dati classici provenienti dall’Erbario Generale dell’Università di Bologna, quali le informazioni contenute nei cartellini e nei relativi documenti.
La ricerca che ho svolto ha verificato la presenza delle specie vegetali identificate nelle regioni di interesse dei siti BeeNet con i campioni presenti su BRAHMS BOLO secondo i criteri già citati nel Cap. \ref{Cap. 2.4}, quali Assenza del campione nell’Erbario Generale (EG-NP), Regione di provenienza del campione (AG), Regioni alle quali richiedere un campione (AGR), Presenza di un solo campione in Erbario Generale (OOSG), Mancato riconoscimento della specie (SD) e Casi particolari (CS). In questo modo è stato possibile definire, per quanto riguarda i parametri di confronto riconducibili all’Erbario Generale: la presenza o meno di campioni di una certa specie nell’erbario generale; confrontare la provenienza regionale della specie identificata con l’origine del campione erborizzato, così definendo a quali regioni richiedere nuovi campioni; verificare la quantità di campioni conservati per singola specie di interesse; definire i mancati riconoscimenti sul campo a livello di specie a causa dell’assenza di determinati caratteri morfologici, identificando gli esemplari soltanto a livello di genere e stabilire la presenza di specie sotto forma di sinonimi o di terminologia scientifica obsoleta.

\end{document}