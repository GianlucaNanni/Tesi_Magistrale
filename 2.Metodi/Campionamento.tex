\documentclass[main.tex]{subfiles}

\begin{document}

\subsection{Campionamento della flora entomofila e relativi pronubi nei siti di studio}

Il network delle api selvatiche si avvale di un monitoraggio che identifica le specie vegetali che si trovano nel transetto, ovvero il percorso predefinito e fisso dove a cadenza mensile avviene il campionamento. Il transetto fisso ha dimensioni di 200x2 metri e la flora ubicata al suo interno è identificata, qualora possibile, a livello di specie mentre per le situazioni dubbie l’identificazione è limitata alla categoria tassonomica di genere. Tutte le specie vegetali in oggetto sono ad impollinazione entomofila ed il riconoscimento è effettuato durante il periodo di antesi, prevalentemente in campo e per mezzo delle conoscenze del relativo personale. In alternativa, le identificazioni possono avvenire in laboratorio tramite fotografie ed annotazioni dei principali caratteri morfologici. In casi particolari i campioni sono prelevati al fine di effettuare l’identificazione tramite flore locali e chiavi dicotomiche \href{https://beenet.crea.gov.it/2022/07/31/come-beenet-osserva-le-api-selvatiche\%EF\%BF\%BC/}{(Come BeeNet osserva le api selvatiche, 2022)}.
Invece, il campionamento degli apoidei selvatici è una metodologia appositamente sviluppata per il Progetto BeeNet e prevede il censimento, tramite cattura con retino, di un campione rappresentativo di pronubi osservati sui fiori all'interno del transetto. Il percorso viene svolto sempre nella medesima direzione ed a passo d’uomo, ha durata di 1 ora e il tempo viene fermato durante le operazioni di sistemazione dell’esemplare catturato. Tali campionamenti sono svolti mensilmente, da marzo a ottobre nelle regioni centro-settentrionali e da febbraio a novembre in quelle meridionali. In ogni giornata vengono eseguiti 2 differenti campionamenti, svolti rispettivamente al mattino e al pomeriggio. Ogni esemplare viene riposto una provetta Falcon contenente truciolato di sughero ed acetato di etile, vengono temporaneamente conservati in una borsa frigo e disposti in congelatore a -20C° una volta giunti in laboratorio. Successivamente gli esemplari catturati vengono spillati ed identificati da un esperto tassonomo.
Le informazioni derivanti dalla flora campionata corrispondenti agli ambienti agricoli a coltivazione intensiva e seminaturale predisposti nelle 11 regioni di riferimento, forniscono importanti dati sulle risorse disponibili per il foraggiamento delle api selvatiche. In tale maniera, oltre all’identificazione delle specie di api selvatiche e piante presenti, è possibile definire le connessioni che avvengono tra insetti e flora in relazione alle caratteristiche agroambientali in cui è ubicato il transetto, consentendo un’analisi della biodiversità del territorio italiano.

\end{document}