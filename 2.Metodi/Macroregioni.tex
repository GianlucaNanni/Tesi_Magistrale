\documentclass[main.tex]{subfiles}

\begin{document}

\subsection{Indice di similarità di Sørensen tra regioni del Nord/Centro/Sud/Isole Italia}\label{Cap. 2.9}

L'indice di Sørensen nella presente analisi viene utilizzato per definire il grado di dissimilarità tra generi vegetali appartenenti rispettivamente ad agroecosistemi intensivi (AI) e seminaturali (ES) delle regioni del Nord, del Centro, del Sud e delle Isole del territorio nazionale.
L’aggregazione dei dati botanici in macroregioni consente di calcolare e comparare, tramite l'indice di Sørensen, quanto i generi tassonomici appartenenti ad agroecosistemi di differenti aree siano dissimili tra loro. La suddivisione delle 11 regioni di interesse in macroregioni avviene secondo l’area geografica di appartenenza (Tab. \ref{tab:3}).

\begin{table}[h!]
    \centering
\begin{tabular}{|1|1|1|}
\hline
Numero di regioni & Regioni & Area\\
\hline
4 & Veneto, Friuli-Venezia Giulia, Emilia-Romagna, Piemonte & Nord\\
\hline
2 & Toscana, Umbria & Centro\\
\hline
3 & Abruzzo, Campania, Puglia & Sud\\
\hline
2 & Sardegna, Sicilia & Isole\\
\hline
\end{tabular}
    \caption{tabella raffigurante l’aggregazione delle regioni in macroaree.}
    \label{tab:3}
\end{table}

In particolare, i siti di riferimento BeeNet utilizzati per il calcolo dell’indice sono così ripartiti all’interno delle macroregioni:

\begin{itemize}
 \item la macroregione NORD è costituita dei seguenti siti: VEAI/VEES (Veneto), FRAI/FRES (Friuli-Venezia Giulia), ERAI/ERESP (Emilia-Romagna) e PIAI/PIES (Piemonte);
  \item la macroregione CENTRO è costituita dei seguenti siti: TOAI/TOES (Toscana) e UMAI/UMES (Umbria);
  \item la macroregione SUD è costituita dei seguenti siti: ABAI/ABES (Abruzzo), CAAI/CAESD (Campania) e PUAI/PUES (Puglia);
  \item la macroregione ISOLE è costituita dei seguenti siti: SAAI/SAES (Sardegna) e SIAI/SIES (Sicilia).
\end{itemize}

Dunque, la somma dei generi vegetali contenuti nei rispettivi siti, intensivi o seminaturali che siano, consente di ottenere i relativi indici di Sørensen utili a definire la somiglianza tra le comunità vegetali delle rispettive macroregioni.
Tali indici statistici sono stati elaborati utilizzando il medesimo software statistico ed i medesimi pacchetti e funzioni citati nel capitolo precedente. (Cap. \ref{Cap. 2.8}).

\end{document}