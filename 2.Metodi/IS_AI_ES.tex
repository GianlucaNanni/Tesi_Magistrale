\documentclass[main.tex]{subfiles}

\begin{document}

\subsection{Indice di similarità di Sørensen tra AI (agroecosistema intensivo) ed ES (agroecosistema seminaturale)}\label{Cap. 2.8}

L'indice di Sørensen misura il grado di somiglianza tra due unità statistiche rappresentate da un set di variabili. Tale indice viene utilizzato per dati derivanti dalla comunità ecologica ed assume un valore compreso tra 0 (massima diversità) e 1 (minima diversità) \citep{biondi}.
Le unità statistiche in oggetto sono rappresentate dai differenti tipi di agroecosistemi intensivi (AI) e seminaturali (ES) delle rispettive regioni di interesse mentre le variabili sono date dal numero delle categorie tassonomiche rilevate.
In particolare, la suddetta analisi considera la somma dei generi tassonomici vegetali identificati in ogni transetto degli agroecosistemi intensivi (AI), appartenenti alle regioni di Abruzzo (ABAI), Campania (CAAI), Emilia-Romagna (ERAI), Friuli-Venezia Giulia (FRAI), Piemonte (PIAI), Puglia (PUAI), Sardegna (SAAI), Sicilia (SIAI), Toscana (TOAI), Umbria (UMAI), Veneto (VEAI). Questo dato viene comparato con la somma dei generi tassonomici vegetali rilevati negli agroecosistemi seminaturali (ES) appartenenti alle regioni di Abruzzo (ABES), Campania (CAESD), Emilia-Romagna (ERESP), Friuli-Venezia Giulia (FRES), Piemonte (PIES), Puglia (PUES), Sardegna (SAES), Sicilia (SIES), Toscana (TOES), Umbria (UMES) e Veneto (VEES) al fine di definire l’indice di similarità di Sørensen. Tale indice viene stabilito per mezzo della formula: 2a/ (2 a + b + c), dove il termine “a” indica il numero di taxa comuni nei rispettivi siti, il termine “b” definisce il numero di taxa esclusivi del primo sito e il termine “c” denota il numero di taxa esclusivi nel secondo sito. Il valore ottenuto dalla comparazione della flora ubicata nei siti intensivi (AI) e seminaturali (ES) fornisce informazioni sul grado di somiglianza delle due comunità.
Inoltre, l’indice di Sørensen viene applicato anche per le specie vegetali identificate nei soli siti intensivi (ERAI) e seminaturali (ERESP) dell’Emilia-Romagna al fine di definire la somiglianza tra le due comunità in oggetto.
Per determinare tale indice ecologico sono stati impiegati in particolare il pacchetto \textit{vegan} \citep{oks} e la funzione \textit{betadiver} \citep{mario}, utilizzando la versione 4.2.0 del Software Statistico R \href{https://www.r-project.org/}{(R Project, 2022)}.

\end{document}