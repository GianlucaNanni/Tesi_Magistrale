\documentclass[main.tex]{subfiles}

\begin{document}

\subsection{Modularità tra le specie di piante e api nei siti ERAI/ERESP}\label{Cap. 2.12}

La modularità è una funzione usata nell'analisi statistica al fine di quantificare la qualità della compartimentalizzazione della rete di moduli, un’idonea suddivisione palesa alti valori di modularità ed è utilizzata per determinare il grado delle interazioni nelle comunità naturali.
Le comunità biologiche e le specie che le costituiscono sono spesso organizzate in network non casuali che palesano modelli ben organizzati e ripetitivi. La modularità ecologica è costituita da moduli (sottoinsiemi) scarsamente interconnessi tra loro ma che al loro interno sono costituiti da specie fortemente connesse. Sia le dimensioni che il numero dei moduli aumentano con il numero di specie.
Indipendentemente dalla numerosità dei gruppi di specie presenti in ogni modulo, le specie svolgono differenti ruoli rispetto alla modularità e i set aventi tratti convergenti possono essere considerati come unità co-evolutive. Dunque, è opportuno identificare e tutelare le specie solidamente collegate all'interno del proprio modulo e quelle che collegano moduli differenti poiché assumono il ruolo di connettori. Le suddette specie che fungono da congiunzione sono da considerarsi elementi chiave ad alta priorità di conservazione, dal momento che a seguito della loro scomparsa i moduli e le relative reti potrebbero interrompersi mettendo a rischio la stabilità del sistema \citep{ole}.
La modularità è stata calcolata per i network di impollinazione dei 2 siti dell’Emilia-Romagna, quali ERAI (Agroecosistema intensivo) ed ERESP (Agroecosistema Seminaturale) e tale valore è stato determinato tramite il Software R e la funzione \textit{likelihood} \citep{goffe}.
Per effettuare tali analisi il software necessita di computare varie volte il network in oggetto e la rete che ne risulta, al termine delle suddette simulazioni, rappresenterà il maggior grado di verosimiglianza. Dunque, assumerà maggiori probabilità di corrispondere alla realtà.

\clearpage

\end{document}