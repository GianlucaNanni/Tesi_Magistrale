\documentclass[main.tex]{subfiles}

\begin{document}

\subsection{Api selvatiche nei siti BeeNet dell’Emilia-Romagna}

Il dataset relativo alle api selvatiche comprende le relazioni mutualistiche che avvengono tra la flora entomofila e gli insetti pronubi nei soli siti di studio dell’Emilia-Romagna (ERAI/ERESP).
Il dataset espresso in tabella (Tab. \ref{tab:4}) comprende: un univoco codice alfa-numerico identificativo che contrassegna la tipologia di agroecosistema in oggetto, la nazione/regione/comune/provincia di interesse, la data del rilievo, le coordinate geografiche WGS84, la specie dell’apoideo identificata e la specie vegetale sulla quale è stata catturata. Qualora la specie pronuba sia catturata soltanto in volo nei pressi della pianta, e non durante una effettiva visita, viene annotata come “in volo”.

\begin{table}[h!]
    \centering
\begin{tabular}[]{|1|~|}
\hline
Id specimen\\
\hline
Country\\
\hline
Region\\
\hline
Municipality\\
\hline
Province Code\\
\hline
Province\\
\hline
Date\\
\hline
Coordinates WGS84 N\\
\hline
Coordinates WGS84 S\\
\hline
Plant\\
\hline
Genus name\\
\hline
Species name\\
\hline
\end{tabular}
    \caption{voci del dataset relativo alle api selvatiche.}
    \label{tab:4}
\end{table}

La circostanza dell’insetto “in volo” è stata omessa dalle analisi delle interazioni piante-impollinatori, in quanto non informativa.

\vspace{6cm}

\end{document}