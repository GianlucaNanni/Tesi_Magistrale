\documentclass[main.tex]{subfiles}

\begin{document}

\subsection{Criteri di selezione per le specie da erborizzare}\label{Cap. 2.4}

In questo capitolo sono illustrati i criteri di classificazione definiti per l’erborizzazione di nuovi campioni, al fine di incrementare la collezione di piante essiccate dell’Università di Bologna. Tali motivazioni sono espresse nei seguenti punti e sintetizzate in sigle:

\begin{itemize}
 \item Assenza del campione nell’Erbario Generale (EG-NP):\\
si intende che la specie vegetale identificata nel determinato sito di rilevamento BeeNet è assente tra i campioni dell’Erbario Generale;
 \item Assenza del campione nell’Erbario Storico (ES-NP):\\
si intende che la specie vegetale identificata nel determinato sito di rilevamento BeeNet è assente tra i campioni dell’Erbario Storico;
 \item Regione di provenienza del campione (AG):\\
si intende che le specie vegetali rilevate nei siti BeeNet hanno una provenienza regionale differente rispetto ai campioni conservati nell’Erbario Generale;
 \item Regioni alle quali richiedere un campione (AGR):\\
è possibile che la specie vegetale sia stata rilevata in più siti di campionamento appartenenti a differenti regioni e che nell’Erbario Generale vi siano più campioni con varie provenienze regionali, in questo modo è definito in maniera diretta le regioni assenti a cui richiedere la determinata specie;
 \item Specie rara (SR):\\
si intende che in base ai criteri IUCN della “Lista Rossa della Flora Italiana - POLICY SPECIES e altre specie minacciate” viene constatato lo stato di rischio di estinzione per le specie rilevate, comprensivo della relativa categoria di minaccia in Italia, quale: Estinta (EX, Extinct), Estinta in natura (EW, Extinct in the Wild), Gravemente minacciata (CR, Critically Endangered), Minacciata (EN, Endangered), Vulnerabile (VU, Vulnerable), Quasi a rischio (NT, Near Threatened), A minor rischio (LC, Least Concern), Dati insufficienti (DD, Data Deficient), Non valutata (NE, Not Evaluated). Soltanto per le valutazioni effettuate a livello sub-globale, ovvero regionale (come le suddette Liste Rosse), vi sono due ulteriori categorie: Estinta a livello regionale (RE, Regionally Extinct) e Non applicabile (NA, Not Applicable) \citep{rossi};
 \item Specie endemica (EN):\\
si intende che in base ai criteri IUCN della “Lista Rossa della Flora Italiana - ENDEMITI e altre specie minacciate” viene constatata la presenza di specie endemiche che vivono unicamente in Italia \citep{rossid};
 \item Presenza di un solo campione in Erbario Generale (OOSG):\\
La constatazione di un solo campione nell’Erbario Generale sottolinea l’importanza di acquisire un numero maggiore di esemplari al fine di implementare il numero di campioni per singola specie. L’acquisizione di campioni provenienti da regioni/zone differenti si identifica come scelta migliore, ma nel caso dovessero provenire dalla medesima regione/zona è da considerarsi ugualmente idoneo per la variabilità dei caratteri;
 \item Presenza di un solo campione in Erbario Storico (OOSS):\\
La presenza di un solo campione nell’Erbario Storico denota l’importanza di tutelare tale esemplare da fenomeni di deterioramento al fine di evitare la perdita di tale patrimonio storico e biologico;
 \item Mancato riconoscimento della specie (SD):\\
durante il processo di riconoscimento l’identificazione è stata effettuata soltanto a livello di genere e non di specie, di fatto in assenza di un univoco nome scientifico non è possibile effettuare alcuna correlazione con i campioni presenti in Erbario;
 \item Casi particolari (CS):\\
La nomenclatura binomiale scientifica delle specie vegetali presenti nell’Erbario Generale e Storico è stata verificata in data 24/05/2022 facendo riferimento al “WFO - The World Flora Online” \href{http://www.worldfloraonline.org/}{(Online Flora of All Known Plants, 2022)} al fine di identificare il corretto nome scientifico, la presenza di sinonimi o binomi non accettati, secondo il codice internazionale di nomenclatura delle piante \citep{turl};
 \item Campione in Erbario Storico deteriorato (CD):\\
Viene verificata l’integrità dei campioni esaminati appartenenti alla Flora Italica di Antonio Bertoloni al fine di determinarne lo stato di conservazione;
 \item Richiesta campione (RC):\\
le specie che appartengono ad uno o più dei sopracitati criteri di selezione sono oggetto di interesse ad essere erborizzate come nuovi campioni, al fine di implementare la collezione di piante essiccate dell’Erbario Generale dell’Università di Bologna.
\end{itemize}

\end{document}