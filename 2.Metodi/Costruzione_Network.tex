\documentclass[main.tex]{subfiles}

\begin{document}

\subsection{Costruzione del Network tra specie vegetali e specie entomologiche}\label{Cap. 2.11}

Il network tra componenti biologiche consente di osservare il quadro delle interazioni ecologiche tra flora entomofila e gli insetti che visitandone i fiori, svolgono il ruolo di potenziali impollinatori. L’interazione tra gli stessi è il punto di partenza per effettuare analisi e grafici utili per definire la struttura delle relazioni esistenti tra le piante ed i relativi pronubi \citep{mara}.
La costruzione e lo studio di tali network sono importanti per comprendere la struttura delle comunità e delle interazioni ecologiche tra piante-impollinatori, al fine di desumere il grado di stabilità del sistema. È possibile valutare le caratteristiche della rete individuando le interazioni specifiche, quelle più specializzate e quelle più generaliste, l’“asimmetria” delle interazioni (o “nestedness”) e la modularità, che costituiscono informazioni importanti in un’ottica di gestione ambientale \citep{bosch}.
I dati oggetto di analisi provengono dai soli siti dell’Emilia-Romagna, quali ERAI (Agroecosistema intensivo) ed ERESP (Agroecosistema Seminaturale), e sono costituiti dagli insetti pronubi identificati e dalla relativa flora entomofila su cui sono stati osservati.
Per effettuare tali indagini statistiche e grafiche mi sono avvalso del Software R, utilizzando il pacchetto \textit{bipartite} \citep{dev} e le relative funzioni per determinare e quantificare le interconnessioni tra i generi vegetali visitati e le specie pronube rilevate.

\end{document}