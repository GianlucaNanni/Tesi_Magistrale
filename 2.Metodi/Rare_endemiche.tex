\documentclass[main.tex]{subfiles}

\begin{document}

\subsection{Considerazioni su specie rare e endemiche}

Le specie vegetali identificate nei siti di riferimento BeeNet sono state comparate con le Liste Rosse Nazionali introdotte dell’Unione Mondiale per la Conservazione della Natura (IUCN) e definite tramite l’applicazione della metodologia IUCN riguardanti le specie rare ed endemiche del nostro Paese.
Le Liste Rosse hanno lo scopo di monitorare e individuare il livello di rischio delle specie al fine di elaborare opportune strategie di intervento per la loro salvaguardia \href{http://www.iucn.it/liste-rosse-italiane.php}{(Liste Rosse italiane, 2022)}.
Le Liste Rosse della Flora Italiana, POLICY SPECIES ed ENDEMITI (cap. \ref{Cap. 2.4}), fanno riferimento alla valutazione della flora a rischio d’estinzione a scala nazionale, determinando che le specie vegetali a maggior rischio sono quelle che vivono in ambienti umidi, soggette a inquinamento, cementificazione e pesticidi. Questi ultimi influenzano pesantemente anche la vita delle api selvatiche, che assolvono una funzione fondamentale per la ricchezza della biodiversità.
Previo controllo dell’aggiornamento nomenclaturale e tassonomico delle specie contenute negli elenchi ho individuato le specie di interesse e relativa Categoria per l’Italia, riconducibili ai criteri: Specie rara (SR) e Specie endemica (EN).

\end{document}