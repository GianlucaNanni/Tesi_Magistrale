\documentclass[main.tex]{subfiles}

\begin{document}

\section{MATERIALI E METODI}
\subsection{Il progetto BeeNet}

Il progetto di ricerca scientifica BeeNet \href{https://beenet.crea.gov.it/il-progetto-beenet/}{(Il progetto BeeNet, 2022)} ha l’obiettivo di monitorare la biodiversità delle api selvatiche su tutto il territorio italiano, al fine di definire le condizioni qualitative dell’agro-ambiente e fornire una visione d’insieme sul mondo delle api e dell’ambiente in cui vivono. La Rete Rurale Nazionale del Ministero delle Politiche Agricole, Alimentari e Forestali è l’ente finanziatore del progetto che viene gestito dal Centro Agricoltura e Ambiente del CREA (Consiglio per la Ricerca in Agricoltura e l’Analisi dell'Economia Agraria).
Nonostante la presente analisi tratti esclusivamente la rete di monitoraggio delle api selvatiche \href{https://beenet.crea.gov.it/rete-api-selvatiche/}{(Biodiversit{\`a} delle api selvatiche, 2022)} è doveroso sottolineare che il Progetto BeeNet si avvale anche di una rete nazionale di monitoraggio per le api mellifere \href{https://beenet.crea.gov.it/rete-api-mellifere/}{(Rete di alveari e arnie, 2022)}. Tale rete oltre a rilevare lo stato di salute delle colonie di api ed i patogeni in grado di danneggiarle, utilizza arnie tecnologiche capaci di rilevare e trasmettere in tempo reale dati sulla colonia. Le informazioni derivanti dai suddetti monitoraggi permettono di definire lo stato di salute dell’agroecosistema nazionale.
La rete di monitoraggio BeeNet è stata attivata sulla base della preesistente rete di monitoraggio ApeNet per ampliare la copertura del servizio sul territorio italiano \citep{lode}. Il duplice scopo che il monitoraggio si prefigge è di creare un sostanzioso database di informazioni sugli apoidei a livello nazionale, così da permettere un confronto diretto, in termini di ricchezza e diversità degli apoidei, tra aree ad agricoltura intensiva ed altre seminaturali in cui l'agricoltura viene praticata in prossimità di aree naturali.
Pertanto, nel progetto BeeNet lo stato di salute delle api non è lo scopo primario dello studio ma lo strumento attraverso il quale viene valutata la condizione dell'agro-ambiente italiano. Tuttavia, dal monitoraggio si otterranno informazioni importanti sulle condizioni degli apoidei selvatici, oltre ad intercettare ed eventualmente prevenire, l'insorgere di eventuali problematiche di tipo patologico, o ambientale.
Il monitoraggio ha dunque la finalità di indicare lo stato di salubrità degli ecosistemi agrari della penisola italiana e di determinare quanto possano essere idonee all’insediamento e alla vita di popolazione di api selvatiche \href{https://beenet.crea.gov.it/2022/03/16/perche-bisogna-controllare-le-api-selvatiche/}{(Controllo delle
api selvatiche, 2022)}. La qualità dell’ambiente agricolo verrà determinato in base allo stato di salute, alla numerosità e all’abbondanza di specie di apoidei selvatici. Inoltre, il database ed i relativi archivi informatici consentono di attuare opportuni sistemi di gestione per le risorse agro-ambientali in oggetto.

\end{document}